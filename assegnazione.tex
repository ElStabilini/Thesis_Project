\documentclass{article}
\usepackage{graphicx} % Required for inserting images
\usepackage{amsmath}
\usepackage{amssymb}
\usepackage[a4paper, total={6in, 8in}]{geometry}


\title{Calibration of superconducting qubits using \tt{Qibocal}}
\author{Elisa Stabilini}
\date{January 2024}

\begin{document}
One of the challenges in gate-based quantum computing is achieving high-fidelity qubits for both single-qubit and two-qubit gate operations. In superconducting qubit platforms, maintaining high fidelity is fundamental for accurately executing quantum circuits and enabling scalable, fault-tolerant quantum computing. To achieve this, qubits must have sufficiently long coherence times to support multiple gate operations, while the implemented quantum gates must minimize errors as much as possible.

The primary goal of the thesis is to develop and implement calibration routines into the Qibocal library to increase the fidelity of both single-qubit and two-qubit gates. 

The first step involves refining an existing calibration to further improve qubit fidelity. 
This can be done by optimizing specific figures of merit for quantum circuits, for example, minimizing the gate infidelity measured through randomized benchmarking. 
In particular, in superconducting qubit systems, gate operations are realized by applying electromagnetic pulses. Subsequently, key tunable parameters for optimization include pulse amplitude and frequency. 

Once fine-tuning is complete, eventual calibration issues can be assessed by implementing routines to correct specific sources of error. For superconducting flux-tunable qubits a possible error source is the distortions in the applied flux pulses that can arise from imperfections in the transmission lines can lead to deviations from the intended pulse shape. To compensate for this effect it is necessary to characterize distortions and then apply pre-distortions which ensures that the applied pulses more accurately correspond to the desired gate operations.

\end{document}