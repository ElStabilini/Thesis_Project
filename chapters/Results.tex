Tutti i risultati che sono presentati nel seguito sono stati ottenuti utilizzando il software di \Qibolab per l'interazione con gli strumenti del laboratorio e \Qibocal per il controllo delle operazioni sui qubit.
L'hardware è un chip ... di QunatumWare.
Durante il lavoro condotto per questo progetto di tesi entrambe le libereria, sia Qibocal che Qibolab undergo update and release, for this reason the first part of this work was realized using \Qibocal v0.1 and \Qibolab v0.1 while the second part of the work, 
dato che puntava anche allo sviluppo di routine ch epotessero essere utili per la calibrazione dei qubit è stato realizzato direttamente con \Qibocal v0.2 e \Qibolab v0.2. 

\section{RB fidelity optimization}

\subsection{Randomized Benchmarking}\label{RBsection}
A strong limitation to the realization of quantum computing technologies is the loss of coherence that happens as a consequence of the application of many sequential quantum gates to to the quibts.
Indeed, a grate challenge faced by quantum computing eexperiments is to physicallly realize gates with low errors whenerver and wherevere applied, currently ... inserire qual è un valore ACCETTABILE.
A possible approach to \textit{gate error charachterization} is the process tomography which allows the experimenter to establish the behaviour of a quantum gates. 
The main drawback of this approach is that process tomography can be very time consumig since its time complexity scales exponentially with the number of qubits involved \cite{QPTomography}

Randomized benchmarking (RB) is technique used to characterize the performance of quantum gates measuring their avarage error rates.
The main idea is that the error obtained from the combined action of random gates applied in sequence to the qubit will avarage out to behave like a depolarizing channel \cite{Emerson_2005_RB}.

It was later shown that simplifies this procedure by restricting the
unitaries to Clifford gates and by not requiring that the sequence is strictly self-inverting

The randomized benchmarking experiment implemented in \Qibocal library, follows a strategy similar to the one described in \cite{knill_randomized_2008}; which consists in the application of random sequences of gates with varying lengths to a given initial state.
The average computational error per gate is then determined by analyzing how the error probability increases in the final measurements as the sequence length grows. 
The random gates are selected from the Clifford group \cite{gottesman1998heisenbergrepresentationquantumcomputers}, which is generated by $\pi/2$ rotations of the form represents a product of Pauli operators acting on the different qubits. 
By restricting the gate set to the Clifford group, the measurements can be performed on single-qubit Pauli operators, ensuring that at least one measurement outcome remains deterministic in the absence of errors.

\paragraph{Haar measure}



\subsubsection{Randomized Benchmarking}
For the results we present in the following the technique used slightly differs from the one described in section \ref{RBsection}, %dato che il tempo richiesto per eseguire una standard RB è dell'ordine di ..
%in qibocal è stata imolementata una variante che si basa sull'implementazione di QUA

\subsection{Optimization methods}
\paragraph{\tt{Optuna}}
\cite{optuna_2019}
\paragraph{\tt{Scipy} methods}
\cite{SciPy-NMeth}
\begin{itemize}
    \item SQLP ?
    \item Nelder-Mead $\rightarrow$ approfondimento
\end{itemize}
\paragraph{CMA - genetics algorithm}
\cite{cmaessimplepractical}

\section{RX90 calibration}

\section{Flux pulse correction}
\subsection{Cryoscope}
\cite{rol_time-domain_2020}
\begin{comment}
    TO DO LIST:
    * calcoli analitic per assunzioni del cryoscope
    * calcoli analitici di convoluzioni per dimostrare che è giusto il modo in cui combiniamo i filtri
    * costruire script per analisi dati
    * eventualmente provare ad aggiungere più correzioni esponenziali
\end{comment}
