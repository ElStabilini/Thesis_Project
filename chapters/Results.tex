\section{RB fidelity optimization}

\subsection{Randomized Benchmarking}
Randomized benchmarking (RB) is technique used to characterize the performance of quantum gates measuring their avarage error rates.
RB was firstly introduced in 2005 \cite{Emerson_2005_RB}, the key idea 

\textbf{disclaimer}: this first study was performed using \tt{qibocal v0.1} the code currently uploaded on thi GitHub repository is instead compatible with \tt{qibocal v0.2}
Main idea: improve fidelity (which one?) fine tuning the calibration

\subsubsection{Randomized Benchmarking}
For the results we present in the following the technique used slightly differ from the 
\paragraph{Randomized Benchmarking on \tt{qua}} %la verisone di Stavros più veloce

\subsection{Optimization methods}
\paragraph{\tt{Optuna}}
\cite{Optuna}
\paragraph{\tt{Scipy} methods}
\cite{Scipy}
\begin{itemize}
    \item SQLP ?
    \item Nelder-Mead $\rightarrow$ approfondimento
\end{itemize}
\paragraph{CMA - genetics algorithm}
\cite{CMA}

\section{RX90 calibration}

\section{Flux pulse correction}
\subsection{Cryoscope}
\cite{Cryoscope}
\begin{comment}
    TO DO LIST:
    * calcoli analitic per assunzioni del cryoscope
    * calcoli analitici di convoluzioni per dimostrare che è giusto il modo in cui combiniamo i filtri
    * costruire script per analisi dati
    * eventualmente provare ad aggiungere più correzioni esponenziali
\end{comment}
