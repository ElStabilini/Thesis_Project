\chapter{RB fidelity optimization}

Tutti i risultati che sono presentati nel seguito sono stati ottenuti utilizzando il software di \Qibolab per l'interazione con gli strumenti del laboratorio e \Qibocal per il controllo delle operazioni sui qubit.
L'hardware è un chip di QunatumWare. %chiedere dettagli sul chip
Durante il lavoro condotto per questo progetto di tesi entrambe le libererie, sia Qibocal che Qibolab undergo update and release, for this reason the first part of this work was realized using \Qibocal v0.1 and \Qibolab v0.1 while the second part of the work, 
dato che puntava anche allo sviluppo di routine ch epotessero essere utili per la calibrazione dei qubit è stato realizzato direttamente con \Qibocal v0.2 e \Qibolab v0.2. 


\section{Randomized Benchmarking}\label{RBsection}
A strong limitation to the realization of quantum computing technologies is the loss of coherence that happens as a consequence of the application of many sequential quantum gates to to the quibts.
A possible approach to characterize gate error is the quantum process tomography which allows the experimenter to establish the behaviour of a quantum gates; the main drawback of this approach is that process tomography can be very time consumig since its time complexity scales exponentially with the number of qubits involved \cite{QPTomography} and the result is affected by state preparation and measurements (SPAM) errors.

To overcome these limitations, randomized benchmarking (RB) was introduced and is currently widely used to quantify the avarage error rate for a set of quantum gates.

The main idea is that the error obtained from the combined action of random unitary gates drawn from a uniform distribution with respect to the Haar measure \cite{Mele_2024} and applied in sequence to the qubit will avarage out to behave like a depolarizing channel \cite{Emerson_2005_RB}.
This last consideration simplifies the characterization of noise because it removes dependence on specific error structures and allows fidelity to be extracted through a simple exponential decay.

It was later shown that it is possible to simplify this procedure even more, by restricting the unitaries to gates in the Clifford group \footnote{unitary rotations mapping the group of Puali operators in itself} and by not requiring that the sequence is strictly self-inverting \cite{knill_randomized_2008}.
\begin{comment}
    The Haar measure is crucial because it provides a unique, invariant way of selecting random elements from a group (in this case, unitary gates) that ensures the randomness is truly uniform and unbiased across the group's structure.
\end{comment}

The fundamental principle of RB is the application of sequences of randomly selected quantum gates from the Clifford group $\mathcal{C}$ followed by an inversion gate which, in absence of noise, return the system to its initial state. 
For real systems, where noise is present, the observed survival probability provides an estimate of the avarage gate fidelity.


The standard RB protocols consist of the following steps:\begin{enumerate}\label{routine:RB}
    \item Initialize the system in ground state $\ket{0}$
    \item For each sequence-length $m$ build a sequence of $m$ randomly drawn Clifford gates $C_1, C_2, ..., C_m$
    \item Determine the inverse gate $C_{m+1}=(C_m\circ...\circ C_1)^{-1}$
    \item Measure $C_{m+1}\circ C_m \circ ...\circ C_1 \ket{0}$
\end{enumerate}
The process must be repeated for multiple sequence of the same length and with varying length.

In ideal systems without noise we should have 
\begin{equation}\label{eq:CliffordIdeal}
    C_{m+1}\circ C_m \circ ...\circ C_1 \ket{0} = (C_m\circ...\circ C_1)^{-1}\circ(C_m\circ...\circ C_1)\ket{0} = \ket{0}
\end{equation}

In real systems, where noise is present, eq. \ref{eq:CliffordIdeal} does not hold; instead randomization with Clifford gates behave as a depolarizing channel \ref{eq:depolarizing_channel} with depolarization probability $d$.
The survival probability of the initial state $\ket{0}$ for different sequence lengths follows the exponential decay model \begin{equation}
    F(m) = Ap^m +B,
\end{equation}
where $1-p$ is the rate of depolarization and $A$ and $B$ capture the state preparation and measurement error but not the rate of decay $p$. 
Note that the exponential form arises naturally due to the assumption that each gate introduces independent noise.

\begin{comment}
    The exponential form arises from modeling each gate application as an independent error process. When you compose multiple noisy gates, their error contributions multiply, leading naturally to an exponential decay.
\end{comment}    

The parameter $p$ is directly related to the depolarization probability $d$ through the avarage gate fidelity $F$ which, for a depolarizing channel, is given by \begin{equation}
    F = 1 - \frac{d}{2^n - 1}\label{eq:avarage_gate_fidelity}.
\end{equation}
For the details of the calculations to obtain eq. \ref{eq:avarage_gate_fidelity} see \hyperref[app:AppendixA]{Appendix A}.

Now we can derive the avarage error per Clifford gate $\epsilon_{Clifford}$ \begin{equation}
    \epsilon_{Clifford} = 1 - F \label{eq:avg_error_Clifford_gate},
\end{equation}
where $F$ is the avarage gate fidelity. Sobstituting in \ref{eq:avg_error_Clifford_gate} the formula for the avarage gate fidelity \ref{eq:avarage_gate_fidelity} we obtain \begin{equation}
    \epsilon_{Clifford} = \frac{d}{2^n -1} = \frac{1-p}{1-2^{-n}},
\end{equation}
which shows how the avarage error per Clifford gate is directly connected to the exponential decay rate.

\subsubsection{QUA Randomized Benchmarking}
For the results we present in the following the technique used slightly differs from the one described in section \ref{RBsection} 
%dato che il tempo richiesto per eseguire una standard RB è dell'ordine di .. la routine su cui sono stari relizzati gli esperimenti è una variante più rapida che è stata implementata sulla base di quanto rirportato in

\section{Choice of parameters to optimize}

\subsection{DRAG technique}

\cite{Motzoi_2009}\cite{Gambetta_2011}


\section{Scipy optimization methods}\label{Sec:OptimizationMethods}

\subsection{Algorithm description}
I primi metodi che abbiamo provato per l'ottimizzazione dei parametri sono quelli standard implementati nella libreria \tt{Scipy} \cite{SciPy-NMeth}.

The first gradient-free optimization method to be tested was Nelder-Mead since in letteratura era già stato riportato il suo utilizzo per obiettivi simili \cite{kelly_optimal_2014}.

The Nelder-Mead optimization method, originally introduced by Nelder and Mead in 1965 \cite{NelderMeads}, is a widely used numerical optimization technique for unconstrained problems in multidimensional spaces. \\
This derivative-free method is operates using simplex, which is a polytope of $n+1$ vertices in a $n$-dimensional space.
The algorithm iteratively updates the simplex by replacing its worst-performing vertex with a new candidate point, thereby guiding the search towards an optimal solution. 
If the goal is to minimize a given function $f(\mathbf{x})$ where $\mathbf{x} \in \mathbb{R}^n$ the algorithms proceeds with the following steps:\begin{enumerate}
    \item If not otherwise initialized, $n+1$ points are sampled for building the initial symplex
    \item \tt{Order} the test points according to their values at vertices: $f(\mathbf{x}_1) \leq f(\mathbf{x}_2) \leq \dots \leq f(\mathbf{x}_{n+1})$ and check whether the algorithm should terminate.
    \item \tt{Calculate} $\mathbf{x}_0$, the centroid of all points except $\mathbf{x}_{n+1}$.
    \item \tt{Reflection}: Compute the reflected point $\mathbf{x}_r = \mathbf{x}_0 + \alpha(\mathbf{x}_0 - \mathbf{x}_{n+1})$ with $\alpha > 0$. 
            If $\mathbf{x}_r$ satisfies $f(\mathbf{x}_1) \leq f(\mathbf{x}_r) < f(\mathbf{x}_n)$, then a new simplex is obtained by replacing the worst-performing point $\mathbf{x}_{n+1}$ with $\mathbf{x}_r$ and then go to step 1.
    \item \tt{Expansion}: If $\mathbf{x_r}$ is the current best point, meaning that $f(\mathbf{x}_r) < f(\mathbf{x}_1)$, then the expanded point is computed: $\mathbf{x}_e = \mathbf{x}_0 + \gamma(\mathbf{x}_r-\mathbf{x}_0)$ with $\gamma>1$.
           If $\mathbf{x}_e$ satisfies $f(\mathbf{x}_e) < f(\mathbf{x}_r)$, then a new simplex is obtained by replacing $\mathbf{x}_{n+1}$ with the expanded point $\mathbf{x}_e$ and then go to step 1.\\
            If instead $f(\mathbf{x}_e) \geq f(\mathbf{x}_r)$, the new simplex is obtained by replacing $\mathbf{x}_{n+1}$ with $\mathbf{x}_r$, and then go to step 1.
    \item \tt{Contraction}: In this case is certain that $f(\mathbf{x}_r) \geq f(\mathbf{x}_n)$ then:\begin{itemize}
        \item If $f(\mathbf{x}_r) < f(\mathbf{x}_{n+1})$: compute the contracted point $\mathbf{x}_c=\mathbf{x}_0 +\rho(\mathbf{x}_{r}-\mathbf{x}_0)$ with $0<\rho \leq 0.5$.
                If $\mathbf{x}_c$ satisfies $f(\mathbf{x}_c) < f(\mathbf{x}_{r})$, then a new simplex is obtained by replacing $\mathbf{x}_{n+1}$ with  $\mathbf{x}_c$ and go to step 1.\\
                Else go to step 6.
        \item  If $f(\mathbf{x}_r) \geq f(\mathbf{x}_{n+1})$: compute the contracted point $\mathbf{x}_c=\mathbf{x}_0 +\rho(\mathbf{x}_{n+1}-\mathbf{x}_0)$ with $0<\rho \leq 0.5$.
                If $\mathbf{x}_c$ satisfies $f(\mathbf{x}_c) < f(\mathbf{x}_{n+1})$, the a new simplex is constructed with $\mathbf{x}_c$ and go to step 1.\\
                Else go to step 6.
    \end{itemize}
    \item \tt{Shrinkage}: Replace all points except the best, $\mathbf{x}_1$, with $\mathbf{x}_i = \sigma(\mathbf{x}_i - \mathbf{x}_1), 0<\sigma \leq 0.5$  
\end{enumerate}
The algorithm terminates when the standard deviation of the function values of the current simplex fall below a user-initialized tolerance. 
When the cycle stops the point of the simplex associated to the lower function value is returned as proposed optimum

The values of the parameters $\alpha, \gamma, \rho$ and $\sigma$ were left to default of \tt{scipy}: $\alpha=1, \gamma=2, \rho=0.5, \sigma=0.5$. 

Per valutare eventuali miglioramenti nella performance abbiamo provato ad utilizzare un algoritmo che fosse gradient-based. 
Nello specifico ho provato ad utilizzare l'algoritmo di Sequential Least Squares Programming (SLSQP) nella versione implementata all'interno della libreria scipy.
%Inserire descrizione algoritmo

\subsection{Results}
Di seguito riporto i risultai che abbiamo ottenuto utilizzando gli algoritmi descritti in precedenza per la minimizzazione del RB.
Per prima cosa ho studiato come l'utilizzo di algoritmi di ottimizzazione che agiscono su ampiezza e durata ... sono in grado di migliorare la assignment fidelity partendo da una calibrazione non ottimalel de qubit 


\section{CMA-ES}

\subsection{Algorithm description}
Covariance Matrix Adaptation Evolution Strategy (\tt{CMA-ES} \cite{cmaessimplepractical}), is a population-based evolutionary algorithm designed for optimizing complex, non-convex, and high-dimensional functions.\\
It belongs to the broader class of Evolution Strategies (ES), a subset of Evolutionary Algorithms (EAs)(see \cite{sloss20192019evolutionaryalgorithmsreview}), and is particularly effective for black-box optimization where gradient information is unavailable.

Evolution Strategies (ES) are a class of optimization methods that employ self-adaptive mechanisms to explore the search space efficiently. 
Unlike classical optimization techniques that rely on gradient descent, ES leverage stochastic sampling to navigate rugged and multimodal landscapes.
In this context, CMA-ES is an adaptive stochastic search method that iteratively refines a probability distribution over the search space. 
Unlike traditional Genetic Algorithms (GAs), which rely on crossover and mutation operators, CMA-ES employs a multivariate normal distribution to generate candidate solutions. 
The method adaptively updates the distribution's mean and covariance matrix based on the fitness of sampled points.

The fundamental idea behind CMA-ES is the use of a multivariate Gaussian distribution to model promising search directions. 
Let $\mathbf{\mu}_t$ denote the mean of the distribution at iteration $t$, and $\Sigma_t$ the covariance matrix. 
Then, a new population of $\lambda$ candidate solutions $\mathbf{x}_i^(t+1) \sim \mathbf{\mu}_t + \sigma_t\mathcal{N}(0, \Sigma_t)$, where $\sigma_t$ is a step size controlling the exploration.  

The CMA-ES algorithm follows the following steps:\begin{enumerate}
    \item If not otherwise specified, the initial parameters are set: mean vector $\mathbf{\mu_0}$, covariance matrix $\Sigma_0$\footnote{$\Sigma_0=\mathbb{I}$ for isotropic search}, step size $\sigma_0$, population size $\lambda$
    \item Generate $\lambda$ new candidate solutions $\mathbf{x}_i$ according to a multivariate normal distribution.
    \item Evaluate the objective function $f(\mathbf{x}_i)$ for each candidate solution.
    \item Sort the new candidate solutions based on fitness: $f(\mathbf{x}_0) \leq ... \leq f(\mathbf{x}_{\lambda})$.
    \item Update the mean vector $\mathbf{\mu}$ with the $m=\lfloor \lambda / 2 \rfloor$ top performing solutions:\begin{equation}
        \mathbf{\mu} \leftarrow \sum_{i=0}^m \mathbf{w}_i\mathbf{x}_1,
    \end{equation} where $\mathbf{w}_i$ are internally defined weights.
    \item Update the isotropic and anisotropic evolution path $\mathbf{p}_{\sigma}$, $\mathbf{p}_c$ \footnote{For details on the update process of the evolution paths see \cite{cmaessimplepractical}.}.
    \item Update the covariance matrix: \begin{equation}
        C \leftarrow (1 - c_1 - c_{\mu}) C + c_1 \mathbf{p}_c \mathbf{p}_c^T + c_{\mu} \sum_{i=1}^{\mu} w_i \mathbf{y}_i \mathbf{y}_i^T,
    \end{equation} where $c_1$ and $c_\mu$ are learning rates and $\mathbf{y}_i$ represents the deviation of the $i$-th cnadidate solution from the mean $\mathbf{mu}$.
    \item Update the step size using a cumulative path evolution mechanism \begin{equation}
        \sigma \leftarrow \sigma \cdot \exp \left( \frac{c_{\sigma}}{d_{\sigma}} \left( \| \mathbf{p}_{\sigma} \| - E \| \mathcal{N}(0, I) \| \right) \right),
    \end{equation} where $c_\sigma$ is the learning rate for step-size adaptation, $d_\sigma$ is a damping factor $\| \mathbf{p}_{\sigma} \|$ is the length of the evolution path and $E \| \mathcal{N}(0, I) \|$ is the expected length of a standard normally distributed random vector.
\end{enumerate}

Nel seguito, a meno che non sia diversamente specificato, i parametri sono stati inizializzati ai valori di default della libereria \tt{CMA-ES}
\subsection{Results}


\section{Optuna}

\subsection{Algorithm description}

In addition to the optimization methods mentioned earlier, the Tree-Structured Parzen Estimator (TPE) method was employed, using its implementation available in the \texttt{optuna} library \cite{optuna_2019}.

Tree-Structured Parzen Estimator (TPE) is a Sequential Model-Based Optimization (SMBO) approach \cite{SMBO_proceedings}. 
SMBO methods sequentially construct models to approximate the performance of optimization parameters based on historical measurements, and then subsequently choose new parameters values to test based on this model. \cite{BayesianOptimizationReview}
At the heart of SMBO is the idea of building a surrogate model, which is used to predict the objective function's values for unseen parameters configurations. 
The surrogate model is iteratively updated as new observations are made, and the optimization process balances exploration, which focuses on uncertain regions of the search space, and exploitation, which focuses on areas that are more likely to improve the objective based on past evaluations. 
This balance ensures that the optimization process makes efficient use of resources and avoids wasting time on suboptimal regions.\\

The TPE algorithm is a probabilistic model-based optimization method that uses non-parametric density estimation to guide the search. 
The TPE algorithm differs from traditional Bayesian optimization approaches, such as Gaussian Process-based methods, in its modeling strategy. 
Rather than directly approximating the objective function, TPE constructs two separate probabilistic models:

\begin{itemize}
    \item $p(x | y < y^*)$, the likelihood of observing a parameter configuration $x$ given that the objective function value $y$ is below a chosen threshold $y^*$.
    \item $p(x | y \geq y^*)$, the likelihood of observing $X$ for less promising function values.
\end{itemize}

These probability densities are estimated using non-parametric methods such as kernel density estimation (KDE). 
New candidate points are then generated by sampling from $p(x | y < y^*)$, favoring configurations that are expected to yield lower objective values. The threshold $y^*$ is typically set as a quantile of observed values, ensuring a focus on the most promising regions of the search space.

The TPE method is the default optimization strategy in \tt{Optuna}, an advantage in the optimization algorithm as implemented in optuna is the addition of an authomatic \textit{pruning} mechanism that stops  unpromising trials early, which can significantly speed up the optimization process by avoiding unnecessary computations.
In our case, this is particularly relevant because the execution of the RB routine, which is performed at each call to the cost function, requires [insert approximate execution time]

As implemented in our code, the default pruner used is the median pruner \tt{optuna.pruners.MedianPruner}. 
This pruner works by evaluating the intermediate results of a trial and comparing them to the median of completed trials at the same step. 
If the current trial's performance is worse than the median, it is pruned to prevent wasting computational resources on unpromising configurations. 

\subsection{Results}

\paragraph{D1 - steps}

\paragraph{D1 - 1000 steps}

\paragraph{D2 - 1000 steps}

\section{RB optimization conclusions}