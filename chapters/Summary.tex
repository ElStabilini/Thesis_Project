\chapter*{Summary}
One of the challenges in gate-based quantum computing is achieving high-fidelity qubits for both single-qubit and two-qubit gate operations.
In superconducting qubit platforms, maintaining high fidelity is fundamental for accurately executing quantum circuits and enabling scalable, fault-tolerant quantum computing. 
To achieve this, qubits must have sufficiently long coherence times to support multiple gate operations, while the implemented quantum gates must minimize errors as much as possible.

During my thesis work, conducted within the Qibo collaboration and specifically in the Qibocal and Qibolab groups, I focused on the study and development of calibration routines for the \Qibocal library aimed at improving both single-qubit and two-qubit gate fidelities.
In the following I will briefly illustrate the content of my thesis report e il lavoro che ho svolto.

%chapter 1 - introduction to quantum computing & superconducting qubits
\paragraph{}


%chapter 2 - experimental setup - software -calibration 
\paragraph{}
Chapter 2 contiene una descrizione dell'apparato sperimentale su cui sono state eseguite tutte le misure e i protocolli svolti durante il lavoro di tesi.
Nello specifico questi sono stati eseguiti sul chip Contralto-D di QuantWare \cite{qw11q} situato al Technology Innovation Institute (TII) di Abu Dhabi. 
Per quanto riguarda l'acquisizione l'elettronica questa era quella fornita da Quantum Machines, nello specifico l'OPX100 \cite{opx1000}.

%chapter 3 . gate optimization
\paragraph{}
The first part of the project addressed the optimization of $R_X(\pi)$ gates, drawing inspiration from the ORBIT technique (Optimized Randomized Benchmarking for Immediate Tune-up) introduced by Kelly et al. in 2014 \cite{kelly_optimal_2014}. 
We explored extensions of this method by testing modern optimization libraries to evaluate whether it is possible to efficiently reach gate fidelities approaching 99.9\%, starting from coarse calibrations, and to assess system stability under realistic drift conditions.

%chapter 4 - pulse correction and control
\paragraph{}
The second part of the work focused on practical needs identified by the experimental teams, specifically the correction of distortions in flux pulses caused by imperfections in the control lines. 
To address this, we implemented the Cryoscope protocol, originally introduced by M. A. Rol in 2019 \cite{rol_time-domain_2020}, which enables the characterization of these distortions and the application of corresponding pre-distortions. This allows for improved accuracy in flux-based gate operations and is now available in Qibocal.

%chapter 5 - conclusions
\paragraph{}

This can be done by optimizing specific figures of merit for quantum circuits, for example, minimizing the gate infidelity measured through randomized benchmarking. 
In particular, in superconducting qubit systems, gate operations are realized by applying electromagnetic pulses. Subsequently, key tunable parameters for optimization include pulse amplitude and frequency. 

Once fine-tuning is complete, eventual calibration issues can be assessed by implementing routines to correct specific sources of error. For superconducting flux-tunable qubits a possible error source is the distortions in the applied flux pulses that can arise from imperfections in the transmission lines can lead to deviations from the intended pulse shape. To compensate for this effect it is necessary to characterize distortions and then apply pre-distortions which ensures that the applied pulses more accurately correspond to the desired gate operations.
