\chapter{Abstract}
Achieving high-fidelity gate operations is a central challenge in gate-based quantum computing. In superconducting qubit platforms, this requires both long coherence times and low-error gate implementations to ensure scalable, fault-tolerant computation.
This thesis, carried out within the Qibo collaboration, specifically in the Qibocal and Qibolab groups, focuses on the development of calibration routines within the \texttt{Qibocal} library, with the aim of improving both single- and two-qubit gate fidelities.

Chapter 1 introduces the theoretical foundation of superconducting qubits, beginning with Josephson junctions and their role in defining the qubit subspace. 
It then reviews early qubit implementations, from Cooper Pair Boxes to flux-tunable transmons, and discusses how these systems support quantum gate operations. 
The chapter concludes with a description of qubit readout mechanisms and sources of decoherence.

In Chapter 2, I describe the experimental setup on which I worked: a superconducting qubit chip developed by QuantWare and operated at the Technology Innovation Institute (TII), using control electronics provided by Quantum Machines. 
The chapter also covers the calibration of single-qubit gates using the routines available in \texttt{Qibocal}. 
These routines were initially used to gain confidence with the calibration process before starting the core of the project. 
The results obtained are summarized in a table reporting the main performance metrics, including readout fidelity, coherence times ($T_1$, $T_2$), and average gate infidelity measured via Randomized Benchmarking.

The work described in Chapter 3 concerns an attempt to automate the recalibration of single-qubit gates using modern optimization libraries. 
In particular, I used tools such as \texttt{Optuna} and \texttt{CMA-ES} to tune key parameters of the $R_X(\pi)$ gate. 
This effort was inspired by the ORBIT protocol (Optimized Randomized Benchmarking for Immediate Tune-up), which relies on Randomized Benchmarking to evaluate fidelity as a cost function. 
While improvements were achieved, the work highlighted several challenges — primarily the high cost of fidelity evaluation and the lack of stability in convergence when broader parameter searches were used.

Chapter 4 describes some enhancements and additions to the calibration infrastructure. 
First, native support for the $R_X(\pi/2)$ gate was introduced in \texttt{Qibolab} and its calibration procedure added to \texttt{Qibocal}, replacing less accurate approximations. 
Second, a protocol for correcting flux pulse distortions was implemented using the Cryoscope method \cite{rol_time-domain_2020}. 
This correction improves the fidelity of flux-based gates such as CNOT and iSWAP, which are particularly sensitive to control-line imperfections.

