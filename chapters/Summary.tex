\chapter*{Summary}
One of the challenges in gate-based quantum computing is achieving high-fidelity qubits for both single-qubit and two-qubit gate operations.
In superconducting qubit platforms, maintaining high fidelity is fundamental for accurately executing quantum circuits and enabling scalable, fault-tolerant quantum computing. 
To achieve this, qubits must have sufficiently long coherence times to support multiple gate operations, while the implemented quantum gates must minimize errors as much as possible.

During my thesis work, conducted within the Qibo collaboration and specifically in the Qibocal and Qibolab groups, I focused on the study and development of calibration routines for the \Qibocal library aimed at improving both single-qubit and two-qubit gate fidelities.
In the following I will briefly illustrate the content of my thesis report e il lavoro che ho svolto.

\paragraph{}
In chapter one I introduce the theoretical framework underlying this work, beginning with a description of Josephson junctions and their operation \cite{JOSEPHSON1962251}. 
These nonlinear electronic elements enable the definition of a two-dimensional subspace that constitutes the qubit. 
The discussion then proceeds to early models of superconducting qubits, from Cooper Pair Boxes (CPBs) \cite{Vion2002} to flux-tunable transmon qubits \cite{TransmonPaper}, and how these systems can be employed to implement single and two qubit gates. 
The final part of the chapter focuses on the mechanisms for qubit state readout and control, concluding with a discussion on the degradation of the qubit state over time.

\paragraph{}
Chapter 2 provides a detailed description of the experimental setup used for all measurements and protocols carried out during this thesis work. 
Specifically, the experiments were performed on the Contralto-D chip developed by QuantWare \cite{qw11q}, hosted at the Technology Innovation Institute (TII) in Abu Dhabi. 
The control and data acquisition electronics were supplied by Quantum Machines, in particular the OPX100 platform \cite{opx1000}.

%chapter 3 . gate optimization
\paragraph{}
The first part of the project addressed the optimization of $R_X(\pi)$ gates, drawing inspiration from the ORBIT technique (Optimized Randomized Benchmarking for Immediate Tune-up) introduced by Kelly et al. in 2014 \cite{kelly_optimal_2014}. 
We explored extensions of this method by testing modern optimization libraries to evaluate whether it is possible to efficiently reach gate fidelities approaching 99.9\%, starting from coarse calibrations, and to assess system stability under realistic drift conditions.

%chapter 4 - pulse correction and control
\paragraph{}
The second part of the work focused on practical needs identified by the experimental teams, specifically the correction of distortions in flux pulses caused by imperfections in the control lines. 
To address this, we implemented the Cryoscope protocol, originally introduced by M. A. Rol in 2019 \cite{rol_time-domain_2020}, which enables the characterization of these distortions and the application of corresponding pre-distortions. This allows for improved accuracy in flux-based gate operations and is now available in Qibocal.

%chapter 5 - conclusions
\paragraph{}
The conclusion chapter summarizes the main results of the work