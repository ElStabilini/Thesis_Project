\chapter*{Summary}
One of the challenges in gate-based quantum computing is achieving high-fidelity qubits for both single-qubit and two-qubit gate operations.
In superconducting qubit platforms, maintaining high fidelity is fundamental for accurately executing quantum circuits and enabling scalable, fault-tolerant quantum computing. 
To achieve this, qubits must have sufficiently long coherence times to support multiple gate operations, while the implemented quantum gates must minimize errors as much as possible.

During my thesis work, conducted within the Qibo collaboration and specifically in the Qibocal and Qibolab groups, I focused on the study and development of calibration routines for the \Qibocal library aimed at improving both single-qubit and two-qubit gate fidelities.
In the following I will briefly illustrate the content of my thesis report e il lavoro che ho svolto.

\paragraph{}
In chapter one I introduce the theoretical framework underlying this work, beginning with a description of Josephson junctions and their operation \cite{JOSEPHSON1962251}. 
These nonlinear electronic elements enable the definition of a two-dimensional subspace that constitutes the qubit. 
The discussion then proceeds to early models of superconducting qubits, from Cooper Pair Boxes (CPBs) \cite{Vion2002} to flux-tunable transmon qubits \cite{TransmonPaper}, and how these systems can be employed to implement single and two qubit gates. 
The final part of the chapter focuses on the mechanisms for qubit state readout and control, concluding with a discussion on the degradation of the qubit state over time.

\paragraph{}
Chapter 2 provides a description of the experimental setup used for all measurements and protocols carried out during this thesis work. 
Specifically, the experiments were performed on the Contralto-D chip developed by QuantWare \cite{qw11q}, hosted at the Technology Innovation Institute (TII) in Abu Dhabi. 
The control and data acquisition electronics were supplied by Quantum Machines, in particular the OPX100 platform \cite{opx1000}.
Moreover, the chapter includes a detailed description of the single gate calibration procedure using the \tt{Qibocal}. 
Learning how to calibrate single qubits and single qubit gates is a prerequisite for implementing optimal control techniques in subsequent stages of the work.

The procedure began with the calibration of the readout resonator, aimed at determining the decoupled resonator frequency and optimizing the amplitude and frequency of the readout pulse. 
This step exploits the dispersive coupling between the resonator and the qubit to enable reliable state discrimination.
Next, the qubit parameters were characterized by identifying its resonance frequency and sweet spot. 
With these parameters, the qubit's ground state configuration could be accurately described.
Subsequently, the focus shifted to the calibration of native single-qubit gates, specifically the RX gate. 
his involved tuning the pulse amplitude and frequency to ensure correct population transfer between the qubit states.
Once these calibrations were completed, a single-shot readout calibration was performed to enable state discrimination in the IQ plane, distinguishing the ground and first excited states.
Finally, fine-tuning experiments were conducted to further correct any residual errors in the RX pulse parameters. 
The chapter concludes with the implementation of randomized benchmarking, a protocol used to quantify gate fidelities by applying sequences of random gates and measuring the decay of state fidelity. This benchmarking routine serves as the foundation for the optimization strategies presented in the following chapter.


%chapter 3 . gate optimization
\paragraph{}
The first part of the project addressed the optimization of avarge Clifford gates fidelity, drawing inspiration from the ORBIT technique (Optimized Randomized Benchmarking for Immediate Tune-up) introduced by Kelly et al. in 2014 \cite{kelly_optimal_2014}. 
We explored extensions of this method by testing modern optimization libraries to evaluate whether it is possible to efficiently reach gate fidelities approaching 99.9\%, starting from coarse calibrations, and to assess system stability under realistic drift conditions.
Nello specifico siamo partiti da una buona calibrazione del device (superiore al 90\%) e poi abbiamo cercato di massimizzare la avarage gate fidelity variando ampiezza e frequenza of the microwave pulse associated with the $R_X(\pi)$ gate e il parametro $\beta$ un parametro moltiplicativo che agisce sulla seconda quadratura dell'impulso di DRAG \cite{Motzoi_2009}.
Indubbiamente questa strategia consente di automatizzare    

%chapter 4 - pulse correction and control
\paragraph{}
The second part of the work focused on practical needs identified by the experimental teams, nello specifico come prima cosa ho aggiunto la $R_X(\pi/2)$ come native gate in \Qibolab e adattato le routine di \Qibocal per la calibrazione della rotazione $R_X(\pi)$ perchè supportassero anche la calibrazione di questa nuova gate.
Prima dell'inserimento di questa nuova gate nel set di native gate assumevamo di poter realizzare una $R_X(\pi/2)$ a partire da una $R_X(\pi)$ semplicemeente by halving the amplitude of the calibrated pi-pulse.
Tuttavia, come mostro nella prima parte di questo capitolo, sono presenti nonlinearities with respect to amplitude in the electronics or any other part of the system. che causano un errore sistematico nella stima dei parametri per l'impulso associato all'$R_X(\pi/2)$.
Il secondo risultato presentato in questo capitolo è l'implementazione di una routine per fare delle correction of distortions in flux pulses caused by imperfections in the control lines. 
To address this, we implemented the Cryoscope protocol, originally introduced by M. A. Rol in 2019 \cite{rol_time-domain_2020}, which enables the characterization of these distortions and the application of corresponding pre-distortions. 
This allows for improved accuracy in flux-based gate operations come mostro ad esempio nelle chevron routine di Qibocal

%chapter 5 - conclusions
\paragraph{}
The conclusion chapter summarizes the main results of the work e possibili outlook / estensioni di questo lavoro.