\section{Qubits}
\section{Operation on qubits}
\subsection{Density matrix}
%Posso rappresentatre l'operatore densità come una matrce 2x2, mi serve per dopo, per la descrizione del depolarizing channel
\section{Quantum operations}
A quantum operation is a mathematical transformation that describes how a quantum state changes as a consequence of a physical process. Formally, it is a map $\mathcal{E}$ that transforms a quantum state described by a density operator $\hat{\rho}$ into another state described by a new density operator $\hat{\rho}'$:
\begin{equation}
    \mathcal{E}(\rho) = \rho'\label{eq:quantum_map}.
\end{equation}

The simplest example of a quantum operation is the evolution of a quantum state $\hat{\rho}$ of a closed quantum system, under a unitary operator $\hat{U}$, which can be written as $\mathcal{E} \equiv \hat{U} \hat{\rho} \hat{U}^{\dagger}$.

\paragraph{Depolarizing chennel}
A depolarizing channel describes a process in which the current state of the $n$-qubit system $\rho$, is replaced by $\frac{\id}{2^n}$, with probability $d$. This process can be represented with a quantum map as follows:
\begin{equation}
    \mathcal{E}_{dc}(\rho) = d\frac{\id}{2^n}+(1-d)\rho\label{eq:depolarizing_channel}
\end{equation} 

\section{Superconducting qubits}

\subsection{cQED systems} \label{sec:cQED}

The frequency $f_Q$ of a two-junction transmon dependends on the magnetic flux $\Phi_Q(t)$ through the SQUID loop, for symmetric junctions is given by\cite{TransmonPaper}
\begin{equation}\label{eq:freqdepndenceonflux}
    f_Q(\Phi_Q) \approx \frac{1}{h} \left( \sqrt{8E_J E_C \cos\left(\pi \frac{\Phi_Q}{\Phi_0} \right)} - E_C \right)    
\end{equation}

where $E_C$ is the charging energy, $E_J$ is the sum of the Josephson energies of the individual Josephson junctions, $\Phi_0$ is the flux quantum, and $h$ is the Planck's constant.