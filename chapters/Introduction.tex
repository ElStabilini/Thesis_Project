\section{Qubits}
\section{Operation on qubits}
\subsection{Density matrix}
%Posso rappresentatre l'operatore densità come una matrce 2x2, mi serve per dopo, per la descrizione del depolarizing channel
\section{Quantum operations}
A quantum operation is a mathematical transformation that describes how a quantum state changes as a consequence of a physical process. Formally, it is a map $\mathcal{E}$ that transforms a quantum state described by a density operator $\hat{\rho}$ into another state described by a new density operator $\hat{\rho}'$:
\begin{equation}
    \mathcal{E}(\hat{\rho}) = \hat{\rho}'\label{quantum_map}.
\end{equation}

The simplest example of a quantum operation is the evolution of a quantum state $\hat{\rho}$ of a closed quantum system, under a unitary operator $\hat{U}$, which can be written as $\mathcal{E} \equiv \hat{U} \hat{\rho} \hat{U}^{\dagger}$.

\paragraph{Depolarizing chennel}

\section{Superconducting qubits}

