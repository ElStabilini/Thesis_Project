\chapter{Pulses analysis and tuning}

Having concluded that the closed-loop optimization protocol we tested would not significantly improve our circuits performance, we shifted our focus towards improvement and implementation of individual protocols to improve the accuracy of qubit operations.\\
In this chapter, I present the results of two additions to the \texttt{Qibocal} software. 
The first is the inclusion of an $RX90$ gate as a native gate, which can enhance the performance of protocols requiring qubit rotations of $\frac{\pi}{2}$.
The second is the implementation of the cryoscope, a routine first described in \cite{rol_time-domain_2020}, which is useful for correcting distortions in the magnetic flux pulse applied to the SQUID.

\section{RX90 calibration}
As discussed in Section \ref{sec:calibration}, it is possible to calibrate system parameters and perform fine-tuning routines to accurately determine the amplitude and frequency of the drive pulse required to transition the qubit from the $\ket{0}$ state to the $\ket{1}$ state. 
This calibration is essential for the correct implementation of single-qubit gates, such as the $R_X(\pi)$ rotation used in our setup.

However, executing more complex quantum circuits and algorithms requires the ability to perform a broader set of quantum operations. 
In gate-based quantum computing, and in particular for superconducting hardware platforms, this is achieved by composing more general unitary operations from a limited set of pre-defined, hardware-native quantum gates. 
These native gates are the elementary operations that are directly implementable and physically calibrated on the quantum processor.

The choice and quality of these native gates are extremly important as all higher-level gates will be decomposed into sequences of native gates, and any calibration errors in the latter can accumulate and propagate through a circuit, degrading the overall fidelity. 
Therefore, having a small, universal, and very well-calibrated set of native gates is essential for the reliable execution of quantum algorithms.

\subsection{\Qibolab native gates}\label{subsec:native_gates}
Native gates are those which can be directly implemented at the hardware level, in contrast to abstract logical gates, which must be transpiled into sequences of hardware supported primitives. 
In the case of \texttt{Qibolab}\footnote{At least for version 0.1 of \Qibolab}, the native gate set consists of $R_X(\pi)$ gate, calibrated be performing the Rabi experiments described in Section \ref{sec:Rabi}, the measurement gate and the virtual-Z (VZ) gate which is not implemented via a physical pulse but rather through a dynamic adjustment of the phase of subsequent control pulses. 
Additionally, the RX90 gate, corresponding to a $\frac{\pi}{2}$ rotation, is implemented by halving the amplitude of the calibrated $\pi$-pulse.
This native gate set, consisting of physically implemented $R_X(\pi)$, $R_X(\pi/2)$, and virtual Z gates, is sufficient for universal single-qubit control.

Indeed, it can be shown that any arbitrary single-qubit unitary operation $ U(\gamma, \theta, \phi) \in SU(2) $ can be expressed, up to a global phase, as a sequence of rotations around the $z$ and $x$ axes of the Bloch sphere:
\begin{equation}
U(\gamma, \theta, \phi) = R_Z(\gamma) R_X(\theta) R_Z(\phi).
\end{equation}
This decomposition ensures that all single-qubit operations can be realized using a combination of $ R_X $ and $ R_Z $ rotations. 

When a qubit is driven by a resonant microwave pulse (with detuning $ \delta = 0 $), the resulting evolution is a rotation around an axis $\hat{n} = (\cos\phi, -\sin\phi, 0)$ lying in the equatorial plane of the Bloch sphere. 
The associated unitary can be written as:
\begin{equation}
R_{\hat{n}(\phi)}(\theta) = \exp\left( -i \frac{\theta}{2} \left[ \cos(\phi)\sigma_x - \sin(\phi)\sigma_y \right] \right).
\end{equation}

This operation can be equivalently expressed through conjugation of an $R_X$ rotation by two $R_Z$ rotations:
\begin{equation}
R_{\hat{n}(\phi)}(\theta) = R_Z(-\phi) R_X(\theta) R_Z(\phi) = U(-\phi, \theta, \phi).
\end{equation}
From this, it follows that any arbitrary unitary $ U(\gamma, \theta, \phi) $ can be implemented as:
\begin{equation}
U(\gamma, \theta, \phi) = R_Z(\gamma + \phi) \cdot R_Z(-\phi) R_X(\theta) R_Z(\phi) = R_Z(\gamma + \phi) \cdot U(-\phi, \theta, \phi).
\end{equation}

Note that, in practice, this final $R_Z$ rotation does not need to be realized as a physical pulse. 
Instead, it can be implemented virtually by adjusting the phase reference of subsequent pulses, a technique known as the virtual-Z gate\cite{McKay_2017}. \\
The ability to compose any single-qubit gate using only $R_X$ and virtual-Z operations \cite{boykin1999universalfaulttolerantquantumcomputing} confirms the sufficiency of this native gate set for universal single-qubit control. For instance, even a restricted $ R_X $ rotation such as $ R_X(\pi) $, when combined with arbitrary-angle $ R_Z $ gates, forms a universal set due to the non-commuting nature of these operations.
By alternating $ R_X(\pi) $ pulses with phase-adjustable $ R_Z $ gates, one can synthesize any $SU(2)$ unitary up to global phase.

\subsection{RX90 as native gate}
Since many routines and protocols in quantum computing rely on $R_X(\pi/2)$ rotations, abbiamo deciso di aggiungere la $R_X(\pi/2)$ gate to the native set of Qibolab.
Il motivo fondamentale è che, fino a questo punto, abbiamo lavoraro che the amplitude or duration of the $R_X(\pi/2)$ gate were exactly half that of the $R_X(\pi)$ pulse.
This assumption implies a perfectly linear response of the qubit to the drive pulse, which may be violated in practice due to nonlinearities in the system or imperfections in the pulse generation and delivery.

Per farlo oltre a inserirla ne [vedi file specifico] I also modified the calibration experiments presented in Section \ref{sec:calibration} per supportare anche la calibrazione di questa nuova gate.




\section{Flux pulse correction}

\subsection{Cryoscope}
The experiment that we describe in this section was first introduced in \cite{rol_time-domain_2020}, the goal is to determine predistortions that needs to be applied to a flux pulse signal so that the qubit receives the flux pulse as intended by the experimenter.

As explained in section \ref{sec:cQED}, accurate dynamical control of qubit frequency is of key importance to realize single- and two-qubit gates.
One of the on-chip control variable that is used on QunatumWare chip is the magnteic flux through a SQUID loop, the signal for magnetic flux control originates from an arbitrary waveform genarator (AWG) which operates at room temperature.\\
As the signal propagates through various electrical components along the control line leading to the quantum device it undergoes linear dynamical distortions. 
If not properly compensated, these distortions can degrade gate performance, impacting experiments fidelity and repeatability.\\

In \cite{rol_time-domain_2020} is proposed a technique to characterize flux-pulse distortions induced by components inside the dilution refrigerator by directly measuring the qubit state.
In this protocol we send the qubit a pulse sequence where a flux pulse of varying duration $\tau$ is embedded between two $\frac{\pi}{2}$ pulses which are always separated by a fixed interval $T_{sep}$.\\
The first $\frac{\pi}{2}$ pulse rotates the qubit of $\frac{\pi}{2}$ around the $y$ axis of the Bloch sphere changing its state from $\ket{0}$ to $\frac{\ket{0}+\ket{1}}{\sqrt{2}}$.

When a flux pulse $\Phi_{Q,\tau}(t)$ of duration $\tau$ is sent to the qubit\footnote{To send a $\Phi_{Q,\tau}(t)$ flux pulse we are actually sending a $V_{\text{in},\tau}(t)$ voltage pulse through the electronics} after the first $\frac{\pi}{2}$ pulse, the qubits evolve to the state $\frac{\ket{0}+e^{i\varphi_\tau}\ket{1}}{\sqrt{2}}$ with relative quantum phase 
\begin{equation}\label{eq:phi}
    \frac{\varphi_{\tau}}{2\pi} = \int_{0}^{T_{sep}} \Delta f_Q (\Phi_{Q,\tau(t)})\text{d}t = \int_{0}^{\tau} \Delta f_Q (\Phi_{Q,\tau(t)})\text{d}t + \int_{\tau}^{T_{sep}} \Delta f_Q (\Phi_{Q,\tau(t)})\text{d}t
\end{equation}
where in the second step we separated the contributions from flux response up to $\tau$ and the turn-off transient. 

The experiment is then completed with a $\frac{\pi}{2}$ rotation aroud the $y$- or $x$-axis of the Bloch sphere to measure respectively the $\langle X \rangle$ or $\langle Y \rangle$ components of the Bloch vector when applying the measurement gate $MZ$. 
From the measurement of $\langle X \rangle$ and $\langle Y \rangle$ we can extract the relative phase $\phi_{\tau}$.\\

Then we can estimate $\Phi_Q(t)$ in the interval $[\tau,\tau+\Delta\tau]$ as follows. From the measurement of $\phi_{\tau + \Delta\tau}$ and $\phi_\tau$ we can compute $\overline{\Delta f_R}$:
\begin{align}\label{eq:detuning}
    \overline{\Delta f_R} &= \frac{\phi_{\tau+\Delta\tau} - \phi_\tau}{2\pi\Delta\tau}\\ 
    &= \frac{1}{\Delta\tau}\left(\int_{0}^{\tau+\Delta\tau}\Delta f_Q (\Phi_{Q,\tau+\Delta\tau}(t))dt + \int_{\tau+\Delta\tau}^{T_{sep}}\Delta f_Q (\Phi_{Q,\tau+\Delta\tau}(t))dt\right) \\
    &-\frac{1}{\Delta\tau}\left(\int_{0}^{\tau}\Delta f_Q (\Phi_{Q,\tau}(t))dt - \int_{\tau}^{T_{sep}}\Delta f_Q (\Phi_{Q,\tau}(t))dt\right)\\
    &=\frac{1}{\Delta\tau}\left(\int_{\tau}^{\tau+\Delta\tau} \Delta f_Q(\Phi_{Q,\tau+\Delta\tau})dt + \int_{\tau+\Delta\tau}^{T_{sep}}\Delta f_Q (\Phi_{Q,\tau+\Delta\tau}(t))dt - \int_{\tau}^{T_{sep}}\Delta f_Q (\Phi_{Q,\tau}(t))dt\right)\\
    &= \frac{1}{\Delta\tau}\int_{\tau}^{\tau+\Delta\tau} \Delta f_Q(\Phi_{Q,\tau+\Delta\tau})dt + \varepsilon
\end{align}  
with \[\varepsilon = \frac{1}{\Delta\tau}\left(\int_{\tau+\Delta\tau}^{T_{sep}}\Delta f_Q (\Phi_{Q,\tau+\Delta\tau}(t))dt - \int_{\tau}^{T_{sep}}\Delta f_Q (\Phi_{Q,\tau}(t))dt\right)\]
The phase contribution from the turn-off transients is minimal due to the sharp return to the first-order flux-insensitive sweet spot of the nearly quadratic $\Delta f_Q(\Phi_Q)$; 
numerical simulations suggest that $|\varepsilon|/\Delta f_R$ remains within the range of approximately $10^{-2}$ to $10^{-3}$ for typical dynamical distortions in commonly used electronic components\cite{negligible}\cite{Langford2017}, for this reason it will be neglected.\\

Then we can obtain the reconstructed flux pulse $\Phi_R(t)$ inverting eq. \ref{eq:freqdepndenceonflux}.
\subsubsection{Pulse reconstruction}

\subsubsection{Corrections study}

\subsection{Corrected pulse}


%TODO: provare ad aggiungere più correzioni esponenziali

\subsection{Filter determination}
\subsubsection{IIR corrections}
\subsubsection{FIR corrections}
Una volta trovati i feedback e feedforward taps per gli IIR filters, è necessario trovare gli FIR per correggere il segnale su scale di tempo più brevi. 

for description and notes on \tt{CMA-ES} see section \ref{Sec:OptimizationMethods}

\subsubsection{Output filters in QM}
%Spiegare come applica i filtri quantum machines
Each analog output port of the OPX system used in this work is equipped with a digital filter that is applied to the signal in the digital domain before conversion to analog. The filters are provided to the OPX via the \texttt{parameters.json} file, which contains the coefficients for the feedforward and feedback components according to the equation
\begin{equation}\label{eq:OPX_filter}
    y[n] = \sum_{m=1}^{M} a_m y[n - m] + \sum_{k=0}^{K} b_k x[n - k],
\end{equation}
where $y[n]$ is the output signal, $x[n]$ is the input waveform, $a_m$ are the feedback coefficients, and $b_k$ are the feedforward coefficients.

In our case, we have a set of feedback coefficients determined through IIR correction and two sets of feedforward coefficients: the first obtained from the IIR-based correction, and the second from FIR-based correction on short timescales. 
To uniquely determine the coefficients to be passed to the electronics, it is necessary to derive a single set of feedforward coefficients and a single set of feedback coefficients by combining the two sets of feedforward filters through convolution.

Infatti posso considerare il segnale di input $x$ a cui applico un primo filtro IIR ottenendo così un segnale in output $y$ tale che
\begin{equation}\label{eq:y_signal}
    y[n] = \sum_{m=1}^{M} a[m]y[n-m] + \sum_{k=0}^{N} b[k] x[n-k].
\end{equation}

al segnale $y$ applico un secondo filtro IIR ottonendo quindi un segnale $z$ in output con la sgeuente forma 
\begin{equation}\label{eq:z_signal}
    z[n] = \sum_{m=1}^{M} a'[m]z[n-m] + \sum_{k=0}^{N} b'[k] y[n-k].
\end{equation}

Now I consider equation \ref{eq:y_signal} and rewrite it as
\begin{equation}\label{eq:y_signal1}
    y[n] - \sum_{m=1}^{M} a[m]y[n-m] = \sum_{k=0}^{N} b[k] x[n-k],
\end{equation} 
then by applying a Z-transform we obtain
\begin{equation}\label{eq:y_signal_transform}
    Y(z)\left(1 - \sum_{m=1}^{M} a[m] z^{-m} \right) = X(z) \left( \sum_{k=0}^{N} b[k] z^{-k} \right)
\end{equation}
so that 
\begin{align}
    H_1(z) &= \frac{Y(z)}{X(z)} = \frac{\sum_{k=0}^{N} b[k] z^{-k}}{1 - \sum_{m=1}^{M} a[m] z^{-m}} = \frac{B(z)}{A(z)} \\
    & \rightarrow Y(z) = H_1(z)X(z) = \frac{B(z)}{A(z)}X(z) \\ \label{eq:transfer1}
\end{align}

We can do the same also for equation \ref{eq:z_signal} and rewrite it as 
\begin{equation}\label{eq:z_signal1}
    z[n] = \sum_{m=1}^{M} a'[m]z[n-m] + \sum_{k=0}^{N} b'[k] y[n-k],
\end{equation}
which, by applying the Z-transform becomes
\begin{equation}\label{eq:z_signal_transform}
    Z(z)\left(1 - \sum_{m=1}^{M} a'[m] z^{-m} \right) = Y(z) \left( \sum_{k=0}^{N} b'[k] z^{-k} \right).
\end{equation}
Again we can write the the transfer function
\begin{align}
    &H_2(z) = \frac{Z(z)}{Y(z)} = \frac{\sum_{k=0}^{N} b'[k] z^{-k}}{1 - \sum_{m=1}^{M} a'[m] z^{-m}} = \frac{B'(z)}{A'(z)} \\
    \rightarrow Z(z) &= H_2(z)Y(z) = \frac{B'(z)}{A'(z)}Y(z) = \frac{B'(z)}{A'(z)} \frac{B(z)}{A(z)} X(z)\\ \label{eq:transfer2}
    &= \left( \frac{\sum_{k=0}^{N} b'[k] z^{-k}}{1 - \sum_{m=1}^{M'} a'[m] z^{-m}} \right) \left( \frac{\sum_{k=0}^{N} b[k] z^{-k}}{1 - \sum_{m=1}^{M} a[m] z^{-m}} \right) X(z)\\
\end{align}
From the transfer function we can obtain the expression for $Z(z)$ in terms of $X(z)$ 
\begin{align}
    & \rightarrow Z(z) \left(1 - \sum_{m=1}^{M} a'[m] z^{-m} \right) \left(1 - \sum_{m=1}^{M} a[m] z^{-m} \right) = \left( \sum_{k=0}^{N} b'[k] z^{-k} \right) \left( \sum_{k=0}^{N} b[k] z^{-k} \right) X(z) \\
    & \rightarrow Z(z) \left( \sum_{m=0}^{M} a'[m] z^{-m} \right) \left( \sum_{m=0}^{M} a[m] z^{-m} \right) = \left( \sum_{k=0}^{N} b'[k] z^{-k} \right) \left( \sum_{k=0}^{N} b[k] z^{-k} \right) X(z)\\ \label{eq:Zz}
\end{align}
where in the last step  ho assorbito l'1 in the sum come coeffiente $a_0$. By expanding the products we obtain
\begin{align}
    & \left( \sum_{m=0}^{M} a'[m] \right) \left( \sum_{m=0}^{M} a[m] \right) = \sum_{m=0}^{2M}\sum_{i=0}^{m} a'[i] a[m - i] = c[k], \quad\quad \text{with}\quad m = 0, \dots, 2M,\\ \label{eq:ck}
    & \left( \sum_{k=0}^{N} b'[k] \right) \left( \sum_{k=0}^{N} b[k] \right) = \sum_{k=0}^{2N} \sum_{i=0}^{k} b[i] b[k-i] = d[k], \quad\quad \text{with}\quad k = 0, \dots, 2N.\\ \label{eq:dk}
\end{align}
%so that
%\begin{align}
%    & \left( \sum_{m=0}^{M} a'[m] z^{-m} \right) \left( \sum_{m=0}^{M} a[m] z^{-m} \right) = \sum_{m=0}^{2M} c[m]z^{-m} = \sum_{m=0}^{2M}\sum_{i=0}^{m} a'[i]a[m - i]z^{-m}\\
%    & \left( \sum_{k=0}^{N} b'[k] z^{-k} \right) \left( \sum_{k=0}^{N} b[k] z^{-k} \right) = \sum_{k=0}^{2N} d[k]z^{-k} = \sum_{k=0}^{2N}\sum_{i=0}^{k} b'[k]b[k-i]z^{-k}.\\
%\end{align}

It is then possible to re-write equation \ref{eq:Zz} using the new expression for the filters
\begin{align}
    & Z(z)\left( \sum_{m=0}^{2M} c[m]z^{-m} \right) = \left( \sum_{k}^{2N} d[k]z^{-k} \right)X(z)\\
    & \rightarrow Z(z)\left(1 - \sum_{m=1}^{2M} c[m]z^{-m} \right) = \left( \sum_{k}^{2N} d[k]z^{-k} \right)X(z)\\ \label{eq:z_final}
\end{align} 

then we apply the inverse-Z-transform and obtain
\begin{align}
    & z[n] - \sum_{m=1}^{2M} c[m]z[n-m] = \sum_{k=0}^{2N} d[k] x[n-k]\\
    & z[n] = \sum_{m=1}^{2M} c[m]z[n-m] + \sum_{k=0}^{2N} d[k] x[n-k]\\
\end{align}
where the feedforward (feedback) taps of the final filters, are given by the convolution of the feedforward (feedback) taps of the two filters as shown in equations \ref{eq:ck} and \ref{eq:dk}.

\subsection{Dimostrazione del conto}

In general, for different forms of the detuning flux $\Delta f(\Phi) = a\Phi^k$, where $k \in \mathbb{Z}^+$, the phase $\varphi_\tau$ expressed in terms of the impulse response $h =\frac{\text{d}s}{\text{d}t}$ is the following,
\begin{align}\label{eq:demonstarted}
    \varphi_\tau &= 2\pi a \int_{0}^{\infty} \left[ \int_{0}^{\infty} h(t - t') dt' - \int_{0}^{\infty} h(t - \tau - t') dt' \right]^k dt\\
     &= 2\pi a \int_{0}^{\tau} \left[ \int_{0}^{t} h(t - t') dt' \right]^k dt + 2\pi a \int_{\tau}^{\infty} \left[ \int_{0}^{\tau} h(t - t') dt' \right]^k dt,
\end{align}

The demonstration of this equality is reported in \hyperref[app:AppendixB]{Appendix B}.

\subsection{Results on chevron pattern}
