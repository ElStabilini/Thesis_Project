\chapter{Pulses analysis and tuning}

Having concluded that closed-loop optimization would not significantly improve fidelity, we shifted our focus towards improvement and implementation of individual protocols to improve the accuracy of qubit operations.\\
In this chapter, I present the results of two additions to the \texttt{Qibocal} software. 
The first is the inclusion of an $RX90$ gate as a native gate, which can enhance the performance of protocols requiring qubit rotations of $\frac{\pi}{2}$.
The second is the implementation of the cryoscope, a routine first described in \cite{rol_time-domain_2020}, which is useful for correcting distortions in the magnetic flux pulse applied to the SQUID.

\section{RX90 calibration}



\section{Flux pulse correction}
\subsection{Notes on signal analysis}


\subsection{Cryoscope}
The experiment that we describe in this section was first introduced in \cite{rol_time-domain_2020}, the goal is to determine predistortions that needs to be applied to a flux pulse signal so that the qubit receives the flux pulse as intended by the experimenter.

As explained in section \ref{sec:cQED}, accurate dynamical control of qubit frequency is of key importance to realize single- and two-qubit gates.
One of the on-chip control variable that is used on QunatumWare chip is the magnteic flux through a SQUID loop, the signal for magnetic flux control originates from an arbitrary waveform genarator (AWG) which operates at room temperature.\\
As the signal propagates through various electrical components along the control line leading to the quantum device it undergoes linear dynamical distortions. 
If not properly compensated, these distortions can degrade gate performance, impacting experiments fidelity and repeatability.\\

In \cite{rol_time-domain_2020} is proposed a technique to characterize flux-pulse distortions induced by components inside the dilution refrigerator by directly measuring the qubit state.
In this protocol we send the qubit a pulse sequence where a flux pulse of varying duration $\tau$ is embedded between two $\frac{\pi}{2}$ pulses which are always separated by a fixed interval $T_{sep}$.\\
The first $\frac{\pi}{2}$ pulse rotates the qubit of $\frac{\pi}{2}$ around the $y$ axis of the Bloch sphere changing its state from $\ket{0}$ to $\frac{\ket{0}+\ket{1}}{\sqrt{2}}$.

When a flux pulse $\Phi_{Q,\tau}(t)$ of duration $\tau$ is sent to the qubit\footnote{To send a $\Phi_{Q,\tau}(t)$ flux pulse we are actually sending a $V_{\text{in},\tau}(t)$ voltage pulse through the electronics} after the first $\frac{\pi}{2}$ pulse, the qubits evolve to the state $\frac{\ket{0}+e^{i\varphi_\tau}\ket{1}}{\sqrt{2}}$ with relative quantum phase 
\begin{equation}\label{eq:phi}
    \frac{\varphi_{\tau}}{2\pi} = \int_{0}^{T_{sep}} \Delta f_Q (\Phi_{Q,\tau(t)})\text{d}t = \int_{0}^{\tau} \Delta f_Q (\Phi_{Q,\tau(t)})\text{d}t + \int_{\tau}^{T_{sep}} \Delta f_Q (\Phi_{Q,\tau(t)})\text{d}t
\end{equation}
where in the second step we separated the contributions from flux response up to $\tau$ and the turn-off transient. 

The experiment is then completed with a $\frac{\pi}{2}$ rotation aroud the $y$- or $x$-axis of the Bloch sphere to measure respectively the $\langle X \rangle$ or $\langle Y \rangle$ components of the Bloch vector when applying the measurement gate $MZ$. 
From the measurement of $\langle X \rangle$ and $\langle Y \rangle$ we can extract the relative phase $\phi_{\tau}$.\\

Then we can estimate $\Phi_Q(t)$ in the interval $[\tau,\tau+\Delta\tau]$ as follows. From the measurement of $\phi_{\tau + \Delta\tau}$ and $\phi_\tau$ we can compute $\overline{\Delta f_R}$:
\begin{align}\label{eq:detuning}
    \overline{\Delta f_R} &= \frac{\phi_{\tau+\Delta\tau} - \phi_\tau}{2\pi\Delta\tau}\\ 
    &= \frac{1}{\Delta\tau}\left(\int_{0}^{\tau+\Delta\tau}\Delta f_Q (\Phi_{Q,\tau+\Delta\tau}(t))dt + \int_{\tau+\Delta\tau}^{T_{sep}}\Delta f_Q (\Phi_{Q,\tau+\Delta\tau}(t))dt\right) \\
    &-\frac{1}{\Delta\tau}\left(\int_{0}^{\tau}\Delta f_Q (\Phi_{Q,\tau}(t))dt - \int_{\tau}^{T_{sep}}\Delta f_Q (\Phi_{Q,\tau}(t))dt\right)\\
    &=\frac{1}{\Delta\tau}\left(\int_{\tau}^{\tau+\Delta\tau} \Delta f_Q(\Phi_{Q,\tau+\Delta\tau})dt + \int_{\tau+\Delta\tau}^{T_{sep}}\Delta f_Q (\Phi_{Q,\tau+\Delta\tau}(t))dt - \int_{\tau}^{T_{sep}}\Delta f_Q (\Phi_{Q,\tau}(t))dt\right)\\
    &= \frac{1}{\Delta\tau}\int_{\tau}^{\tau+\Delta\tau} \Delta f_Q(\Phi_{Q,\tau+\Delta\tau})dt + \varepsilon
\end{align}  
with \[\varepsilon = \frac{1}{\Delta\tau}\left(\int_{\tau+\Delta\tau}^{T_{sep}}\Delta f_Q (\Phi_{Q,\tau+\Delta\tau}(t))dt - \int_{\tau}^{T_{sep}}\Delta f_Q (\Phi_{Q,\tau}(t))dt\right)\]
The phase contribution from the turn-off transients is minimal due to the sharp return to the first-order flux-insensitive sweet spot of the nearly quadratic $\Delta f_Q(\Phi_Q)$; 
numerical simulations suggest that $|\varepsilon|/\Delta f_R$ remains within the range of approximately $10^{-2}$ to $10^{-3}$ for typical dynamical distortions in commonly used electronic components\cite{negligible}\cite{Langford2017}, for this reason it will be neglected.\\

Then we can obtain the reconstructed flux pulse $\Phi_R(t)$ inverting eq. \ref{eq:freqdepndenceonflux}.
\subsubsection{Pulse reconstruction}

\subsubsection{Corrections study}

\subsection{Corrected pulse}

\begin{comment}
    TO DO LIST:
    * calcoli analitici di convoluzioni per dimostrare che è giusto il modo in cui combiniamo i filtri
    * eventualmente provare ad aggiungere più correzioni esponenziali
\end{comment}
\subsection{Filter determination}
\subsubsection{IIR}
\subsubsection{FIR}
for description and notes on \tt{CMA-ES} see section \ref{Sec:OptimizationMethods}
\subsubsection{Output filters in QM}

\subsection{dimostrazione del conto}

In general, for different forms of the detuning flux $\Delta f(\Phi) = a\Phi^k$, where $k \in \mathbb{Z}^+$, the phase $\varphi_\tau$ expressed in terms of the impulse response $h =\frac{\text{d}s}{\text{d}t}$ is the following,
\begin{align}\label{eq:demonstarted}
    \varphi_\tau &= 2\pi a \int_{0}^{\infty} \left[ \int_{0}^{\infty} h(t - t') dt' - \int_{0}^{\infty} h(t - \tau - t') dt' \right]^k dt\\
     &= 2\pi a \int_{0}^{\tau} \left[ \int_{0}^{t} h(t - t') dt' \right]^k dt + 2\pi a \int_{\tau}^{\infty} \left[ \int_{0}^{\tau} h(t - t') dt' \right]^k dt,
\end{align}

The demonstration of this equality is reported in \hyperref[app:AppendixB]{Appendix B}.