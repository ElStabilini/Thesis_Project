Desidero innanzitutto ringraziare il professor Carrazza per avermi offerto l'opportunità di svolgere questa tesi all'interno del progetto Qibo e per il supporto fornito durante tutte le fasi del lavoro.

Un sentito ringraziamento va anche ai miei correlatori per la guida e il costante supporto. La loro disponibilità e il loro approccio aperto sono stati fondamentali per affrontare con maggiore consapevolezza il percorso di ricerca.\\
Ringrazio Alessandro per l'aiuto e i consigli ricevuti, in particolare durante le settimane trascorse al TII. La sua attenzione agli aspetti umani del lavoro di ricerca e la sua disponibilità hanno rappresentato un punto di riferimento importante.\\
Ringrazio Edoardo per la costante disponibilità, in particolare nelle fasi iniziali del progetto, durante le quali mi ha supportata nella comprensione della struttura e del funzionamento di \tt{qibocal} e \tt{qibolab}. Il suo aiuto è stato essenziale per orientarmi all'interno del codice.\\
Ringrazio inoltre Andrea per la disponibilità e il supporto puntuale, che si sono rivelati particolarmente utili ogni volta che ho avuto bisogno di chiarire aspetti complessi o approfondire passaggi tecnici.
\paragraph{}
Ringrazio i miei genitori che hanno reso possibile questo percorso, senza opporsi alle scelte che non condividevano o non capivano ma fidandosi della mia capacità di scegliere ciò che era meglio per me.
Soprattutto ringrazio mia mamma Raffaella da cui ho imparato a impegnarmi nel mio lavoro e fare del mio meglio, e ringrazio mio papà che mi ha insegnato a non spaventarmi di fronte a qualcosa che non capisco ma a farmi domande e cercare di capire in profondità il funzionamento di quello che ho davanti.
Ovviamente ringrazio mia sorella Anna, la mia cincips, che mi ha sempre sostenuto, spronato ad andare avanti e spinta a continuare anche quando non sono stata bene, anche quando le cose sembravano non andare e mi ha dato la forza di continuare passo dopo passo, è bello averti accanto.
Ti ringrazio anche per tutte le foto di Wendy e Mila che mi ha mandato quando io non ero a casa ma tu sì e che mi hanno sempre strappato un sorriso o una lacrima per la tenerezza.
\paragraph{}
Ringrazio la nonna Palmira che mi ha insegnato a perseguire i miei obiettivi con ostinazione senza mai rinunciare ad aiutare chi incontravo nel mio percorso, e che mi ha sempre ripetuto che sarebbe andato tutto bene, questa tesi è dedicata anche a te.
Ringrazio la nonna Rina che mi ha sempre ripetuto .
Ringrazio anche i nonni, il nonno Giuseppe e il nonno Mario che porto entrambi dentro di me, questo traguardo l'ho raggiunto anche grazie a voi che avete sempre fatto il massimo per la famiglia.
\paragraph{}
Ringrazio tutti i miei compagni di corso con cui ho condiviso il mio percorso fino a questo punto, è stato bello incontrarsi e poter condividere tanti momenti che hanno reso piacevole il tempo trascorso in università.
Ringrazio Tima per tutti gli abbracci che mi ha regalato, Paolo per i le risate e il caffè la mattina, perché cominciare la giornata insieme rende tutto un po' più facile, Ale per la carica di passione e voglia di fare che mi trasmette ogni volta che facciamo una chiacchierata.\\
Ringrazio le mie compagne storiche del gruppo del laboratorio, Rosa e Daniela perché cominciare questo percorso con voi ha decisamente alleggerito le difficoltà, e fatto sembrare affrontabili anche i momenti meno semplici. 
Ringrazio soprattutto Alessia, in questi cinque, ormai sei, anni in cui ci conosciamo la nostra amicizia si è evoluta moltissimo, sei stata un'ispirazione, con tutta la tua energia, i tuoi progetti, la tua positività e la tua simpatia, sei anche stata una spalla a cui mi sono potuta appoggiare ogni volta che ne ho avuto bisogno.
Sono molto contenta di poter condividere con te tutti i miei pensieri e le mie idee sapendo che troverò sempre una persona pronta ad ascoltarmi e dirmi con onestà che cosa pensa senza mai giudicarmi.
Ringrazio qui anche Ire, che conosco da ormai tre anni e con cui ho avuto modo di confidarmi e confrontarmi, sapere di avere il tuo supporto, anche a distanza significa molto.
\paragraph{}
Un ringraziamento speciale è anche per tutte le persone che ho incontrato durante quella \textit{vacanza} meravigliosa che è stato il tirocinio al Fermilab, ognuno in modo diverso mi avete aiutato tutti a crescere.
Sicuramente ringrazio Tom per tutti i consigli che mi ha sempre dato, Marco che non ha mai paura di condividere le sue idee, di cui è sempre stato aperto a discutere con me e che durante i due mesi trascorsi insieme ci ha coinvolti e trascinati tutti in gite bellissime.
Ringrazio il mio compagno d'ufficio, Ste, in quei due mesi abbiamo condiviso tanto, dai gossippini alle paure e difficoltà, che fossero legate all'università o alla vita. 
Ringrazio moltissimo Sara e Silvia, grazie a voi ho sinceramente scoperto il valore di avere accanto due donne incredibili che ti sostengono e ti spronano a fare il tuo meglio, la nostra amicizia mi ha fatto scoprire quanto incredibile possa essere l'alleanza tra donne. 
Siete state le mie prime confidenti per qualsiasi problema io abbia avuto durante i due mesi in cui abbiamo vissuto insieme, siete state una sorpresa fantastica di cui non sapevo nemmeno di avere bisogno.
Come ci siamo già dette molte volte, vivere insieme ad altre due sconosciute avrebbe potuto essere difficile e invece l'avete resa la cosa più bella e naturale che si possa immaginare, non vedo l'ora di poter vedere tutta la strada che sono sicura percorrerete, ognuna nel suo ambito.
\paragraph{}
Ringrazio tutte le persone che non ho nominato ma che mi sono state accanto, il supporto di tutte è stato fondamentale!
