\begin{comment}
    Carrazza
    Alessandro, Edoardo, Andrea
    Altri collaboratori di Qibo
    Alessia
    Sara & Silvia
    Tutti amici uni
    Anna, genitori nonna
\end{comment}

In primo luogo non posso che ringraziare il professor Carrazza per avermi consentito di fare questo lavoro, per il supporto che mi ha fornito durante lo svolgimento del lavoro e soprattutto per la fiducia che mi ha accordato a scatola chiusa per poter fare questa tesi.
Sicuramente ringrazio i miei correlatori senza i quali non avrei saputo da che parte cominciare, li ringrazio per avermi guidato senza mai impormi delle scelte e per non essersi mai scandalizzati per la mia ignoranza, è stato fondamentale.
Ringrazio Alessandro perchè soprattutto nelle due settimane passate al TII mi hai aiutata a inserirmi nel progetto, a capire come fosse meglio lavorare e sei sempre stato pronto a darmi consigli per migliorare. Mi ha aiutato molto la tua disponibilità e la tua attenzione al lato umano della ricerca, non immaginavo che avrei potuto trovare anche questo nel mondo della ricerca. 
Ringrazio Edoardo per essere stato sempre disponibile, soprattutto nel primo periodo quando stavo cercando di capire in quali parti del codice potessi mettere mano (e come mettere mano) non avrei potuto fare altrimenti.
Ringrazio anche Andrea per essere stato sempre disponibile non appena ne ho avuto bisogno e per avermi aiutato tantissimo con il cryoscope senza farmi sentire stupida.
Ringrazio tutto il gruppo di Qibo, è stato bello sentirsi parte di un progetto e mi avete accolta senza mai farmi sentire a disagio o non all'altezza di quello che dovevo fare.
Ringrazio i miei genitori che hanno reso possibile questo percorso, senza opporsi alle scelte che non condividevano o non capivano ma fidandosi della mia capacità di orientarmi.
Ringrazio le nonne, la nonna Rina e la nonna Palmira, che sono state una presenza rassicurante e stimolante, con sempre tanto affetto nei miei confronti. 
Ringrazio anche i nonni, il nonno Giuseppe e il nonno Mario che oggi non sono qui ma dai cui racconti ho imparato tanto e di cui sicuramente porto un pezzo dentro di me.
Ci tengo moltissimo a ringraziare mia sorella Anna, la mia cincips, che mi ha sempre sostenuto, spronato ad andare avanti e spinta a continuare anche quando non sono stata bene, anche quando le cose sembravano non andare e mi ha dato la forza di continuare passo dopo passo.
La ringrazio anche per tutte le foto di Wendy e Mila che mi ha mandato quando io non ero a casa ma lei sì e che mi hanno sempre strappato un sorriso o una lacrima per la tenerezza.
Ringrazio tutti i miei compagni di corso con cui ho condiviso questo percorso fino a questo punto, le nostre strade probabilmente si divideranno ma è stato bello incontrarsi e poter condividere tanti momenti, spero che le nostr strade si incontrino di nuovo.
Ringrazio Tima per tutti gli abbracci che mi ha regalato, Paolo per i le risate e il caffè la mattina, perchè cominciare la giornata insieme rende tutto un po' più facile, Ale per la carica di passione e voglia di fare che mi trasmette ogni volta che facciamo una chiacchierata.
Ringrazio le mie compagne storiche del gruppo del laboratorio, Rosa e Daniela perchè studiare e confrontarmi con voi è stato incredibilmente arrichhente, ha alleggerito le difficoltà, e fatto sembrare anche i momenti meno semplici comunque affrontabili. 
Ringrazio soprattutto Alessia, in questi cinque anni, ormai sei, in cui ci conosciamo la nostra amicizia si è evoluta moltissimo, sei stata un'ispirazione, con tutta la tua energia, i tuoi progetti, la tua positività e la tua simpatia, sei anche stata una spalla a cui mi sono potuta appoggiare ogni volta che ne ho avuto bisogno,
sia per quanto riguarda tutto ciò che è successo nella mia vita in questi cinque anni che per le cose successe in università. Sono molto contenta di poter condividere con te tutti i imiei pensieri e le mie idee sapendo che troverò sempre una persona pronta ad ascoltarmi e dirmi con onestà che cosa pensa senza mai giudicarmi.
Un ringraziamento speciale è anche per tutte le persone che ho incontrato durante quella \textit{vacanza} mervigliosa che è stato il tirocinio al Fermilab, ognuno in modo diverso mi avete aiutato tutti a crescere.
Sicuramente ringrazio Tom per tutti i consigli che mi ha sempre dato, Marco che non ha mai paura di condividere le sue idee, molte in comune altre no ma su cui è sempre stato aperto a discutere, per non parlare della sua propensione alle avventure e al vivere sempre al massimo che ci ha coinvolto e caricati tutti durante i due mesi passati insieme.
Ringrazionì moltissimo Sara e Silvia, grazie a voi ho sinceramente scoperto il valore di avere accanto due professioniste incredibili che ti sostengono e ti spronano a fare il tuo meglio, la nostra amicizia per me è speciale. Siete state le mie prime confidenti per qualsiasi problema io abbia avuto durante i due mesi in cui abbiamo vissuto insieme, sietet state una sorpresa fantastica di cui non sapevo nemmeno di avere bisogno.
Come ci siamo già dette molte volte, vivere insieme ad altre due sconosciute avrebbe potuto essere difficile e invece l'avete resa la cosa più bella e naturale che si possa immaginare, non vedo l'ora di poter vedere tutta la strada che percorrerete.

