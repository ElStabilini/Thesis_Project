\chapter{Conclusions and outlook}

In this thesis, after introducing the main theoretical concepts and the hardware and software tools employed, the focus shifted to the calibration procedures for single qubits and their optimization. 
The theoretical framework was focused on superconducting quantum devices, which constitute the experimental platform used throughout the work.

\paragraph{Calibration experiments}
As a preliminary step for this work, it was essential to gain practical experience with the calibration procedures required for operating a superconducting qubit. 
For this reason, the second Chapter of the thesis is devoted to describing in detail the full calibration workflow for a single-qubit device. 
In addition to outlining the experimental steps, I also aimed to provide basic explanations of the underlying physical principles, particularly those from quantum mechanics and circuit quantum electrodynamics (cQED), that enable control and readout of superconducting qubits. 

Becoming familiar with the reasoning behind each calibration step was important for effectively working with both the hardware and the software tools used in the project.
To support this understanding, I carried out a full calibration of a specific qubit line, namely the D-line of the \tt{qw11q} chip. 
The results of these calibration procedures are summarized in Table \ref{tab:cal_results}, and served as the foundation for the subsequent experiments focused on gate fidelity optimization.

\paragraph{RB optimization}
In Chapter 3, I presented the results of an experimental study exploring the feasibility of automated optimization routines for improving the average fidelity of single-qubit Clifford gates, using Randomized Benchmarking (RB) as the primary figure of merit. 
The results show that, despite several technical challenges, such routines can yield measurable improvements in fidelity.

From a performance standpoint, the Nelder-Mead simplex algorithm demonstrated the most reliable convergence, although at the cost of high resource consumption due to the large number of RB evaluations required at each step. 
In contrast, \tt{Optuna} allowed for efficient and broad exploration of the parameter space within a more limited runtime, especially when the number of trials was constrained. 
However, the optimization outcomes were often unstable: small changes in control parameters could lead to significant variations in the resulting avarage Clifford gate fidelity. 
This behavior highlights a limitation in the current approach, namely a possible lack of robustness in the system's response to parameter tuning due to parameters drift.

The main drawback of this optimization strategy lies in the time cost associated with evaluating the fidelity via RB. 
Each iteration of the optimization routine involves executing an RB sequence, making the overall process time-consuming. 
This is a substantial limitation, especially considering the broader applicability of RB as a fidelity estimator across different quantum platforms, including those with slower gate execution times than superconducting qubits.

To address this bottleneck, a natural direction for future work is to try optimize the RB routine itself. 
Reducing the depth of the Clifford sequences or the number of circuit samples used per evaluation could help shorten runtime without significantly sacrificing accuracy. 
This could be achieved by performing a systematic hyperparameter optimization of the RB settings. 

Despite the current limitations, automating calibration remains an essential step toward scalable quantum control. 
Without such routines, calibration would rely on manual tuning, which involves iteratively guessing suitable parameter configurations; a slow and impractical approach, especially as systems grow in size and complexity. 
In summary, although the system is not yet ready for deployment as a fully reliable calibration tool, this work provides an initial basis for exploring automated, hardware-level optimization procedures that could contribute to improving gate fidelity in a scalable and maintainable way.


\paragraph{Calibration routines}
In Chapter 4 of this work I presented the results of practical improvements to the calibration framework based on challenges and requirements that emerged from experimental operations. 
Specifically, we introduced a set of enhancements to the \Qibolab and \Qibocal libraries to support the $R_X(\pi/2)$ gate as a native operation. 
This involved both modifying the back-end to natively implement the gate and updating the associated calibration routines, which were originally designed only for the $R_X(\pi)$ gate.
The inclusion of $R_X(\pi/2)$ as a native gate is important as such $\pi/2$ rotations are frequently used in quantum algorithms and benchmarking protocols. 
An inaccurate calibration would propagate systematic errors throughout the execution of quantum circuits, ultimately degrading their overall fidelity.

The chapter also introduces the implementation of the Cryoscope protocol to correct for dynamical distortions in the control lines used to apply flux pulses. 
In superconducting qubit platforms, accurate frequency control via flux modulation is fundamental for high-fidelity single- and two-qubit gates. 
However, as the flux signal propagates through the various components of the control line, it is subject to distortions resulting from finite bandwidth, reflections, and impedance mismatches. 
These effects introduce deviations between the intended and actual qubit trajectories, which in turn affect gate performance. 
The Cryoscope protocol enables a detailed, time-domain characterization of these distortions, providing the basis for computing pre-distorted waveforms that effectively cancel out the unwanted dynamics. 
The improvements obtained using these corrected pulses are particularly evident in the chevron calibration routine in \Qibocal, which plays a central role in tuning flux-based two-qubit gates such as CNOT and iSWAP.

To further enhance the fidelity of flux pulses, additional corrective strategies can be employed. 
These include techniques designed to minimize leakage outside the computational subspace, compensate for pulse echoes resulting from impedance mismatches, and suppress residual ringing that follows sharp transitions in the control waveform. 
Other approaches focus on more precise correction of long-time distortions, such as the method described in the supplementary material of \cite{Ma_2019}. 
Collectively, these improvements contribute to higher gate fidelity, increased pulse robustness, and more accurate temporal control of qubit dynamics.