% % -*- coding:utf-8 -*-
\documentclass[aspectratio=169,10pt]{beamer}
\nonstopmode

\usepackage{appendixnumberbeamer}
\usepackage{graphicx}
\usepackage{url}
\usepackage{cite}
\usepackage{comment}
%\usepackage{verbatim}
\usepackage{enumitem}
\usepackage{caption}
\usepackage{tikz}

\captionsetup{skip=4pt}
\input{colors}

% \usepackage{beamerthememetropolis}
\usetheme[progressbar=frametitle]{metropolis}
\metroset{progressbar=none}
\newcommand{\themename}{\textbf{\textsc{metropolis}}\xspace}


\usepackage{xcolor}
% Remove default navigation symbols
\setbeamertemplate{navigation symbols}{}
\setbeamercolor{background canvas}{bg=white}

% Custom footline with frame numbers on the left and navigation symbols on the right
\setbeamertemplate{footline}{
  \leavevmode%
  \hbox{%
    \begin{beamercolorbox}[wd=.1\paperwidth,ht=2.5ex,dp=1.125ex,leftskip=1em,rightskip=1em]{author in head/foot}%
      \usebeamerfont{footline} \insertframenumber{} / \inserttotalframenumber
    \end{beamercolorbox}%
    \hfill
    \begin{beamercolorbox}[wd=.9\paperwidth,ht=2.5ex,dp=1.125ex,leftskip=1em,rightskip=1em]{author in head/foot}%
      \usebeamerfont{footline} %\insertnavigation{2cm}
    \end{beamercolorbox}%
  }%
  \vskip0pt%
}


\title{Development of an open-source calibration framework\\ for superconducting qubits}
\subtitle{Master degree in Physics}
\author{Elisa Stabilini}
\institute{Università degli Studi di Milano - Department of Physics}
\titlegraphic{\hfill}
\date{July 4th 2025}

\begin{document}

\maketitle

\begin{frame}{Table of contents}
    \setbeamertemplate{section in toc}[sections numbered]
    \setbeamertemplate{subsection in toc}[subsections numbered]  
    \tableofcontents[hideallsubsections]
\end{frame}

\begin{frame}[t,fragile]{Qibo framework}
  \begin{center}
      \includegraphics[height=0.80\paperheight]{figures/qibo_ecosystem.png}
  \end{center}
\end{frame}

\section{Superconducting qubits}

\begin{frame}[t,fragile]{Artificial atoms}
  \begin{columns}
    \begin{column}{0.5\textwidth}
      \begin{itemize}
        \item Qubit: two level system
        \item Superconducting qubits: use Josephson Junctions to build anharmonic oscillators
      \end{itemize}
    \end{column}
    \begin{column}{0.5\textwidth}
      
    \end{column}
  \end{columns}
%Josephson junction
\end{frame}

%
\begin{frame}[t,fragile]{State readout}
  \begin{columns}
    \begin{column}{0.35\textwidth}
    Qubit - resonator hamiltonian:
    \begin{equation*}
      \hat{H} = \hbar\omega_r\hat{a}\hat{a}^\dagger - \frac{\hbar\omega_{01}}{2}\hat{\sigma}_z + \hbar g(\hat{\sigma}^+\hat{a}+\hat{\sigma}^-\hat{a}^\dagger)
    \end{equation*}
    Dispersive regime $g \ll \omega_q - \omega_r$
    \begin{equation*}
      \hat{H}_{disp} = \hbar(\omega_r - \chi\hat{\sigma}_z)\hat{a}^\dagger\hat{a} - \frac{\hbar}{2}(\omega_{01}+\chi)\hat{\sigma}_z
    \end{equation*}
    dispersive shift: $\chi = \frac{g^2}{\Delta}, \Delta = \omega_q - \omega_r$
  \end{column}
    \begin{column}{0.65\textwidth}
      \begin{center}
        \begin{figure}
          \vspace{2mm}
          \includegraphics[width=0.85\textwidth]{figures/TransmonCircuit.png}
          \vfill
          \includegraphics[width=0.85\textwidth]{figures/TransmonBoard.png}
          \caption{DOI: 10.1109/MAP.2022.3176593}
        \end{figure}
      \end{center}
    \end{column}
  \end{columns}
\end{frame}
%
%\begin{frame}[t,fragile]{Calibration}
%  
%\end{frame}

\section{Average Clifford gate fidelity optimization}

\begin{frame}[t,fragile]{Randomized Benchmarking}
\begin{center}
  Randomized benchmarking estimates average gate fidelity by applying random sequences of Clifford gates followed by an inverting gate.
  \vspace{3mm}
  \begin{figure}
      \includegraphics[height=0.52\textheight]{figures/RBcircuit.png}
      \caption{DOI: 10.1007/s10773-024-05811-8}
  \end{figure}
\end{center}
\end{frame}

\begin{frame}[t,fragile]{Randomized Benchmarking}
\begin{center}
  Randomized benchmarking estimates average gate fidelity by applying random sequences of Clifford gates followed by an inverting gate.
  \vfill
  \includegraphics[width=\textwidth]{figures/rb.png}
\end{center}
\end{frame}

\begin{frame}[t,fragile]{RB optimization}%insert listings 
%\begin{center}
%  Randomized benchmarking estimates average gate fidelity by applying random sequences of Clifford gates followed by an inverting gate.
%  \vfill
%  \includegraphics[width=0.6\texwidth]{figures/rb.png}
%\end{center}
\end{frame}


%\begin{frame}[t,fragile]{RB optimization results}
%  
%\end{frame}

\section{Library additions}

\subsection{Native RX90}

%\begin{frame}[t,fragile]{Native RX90 gate}
%  
%\end{frame}

\subsection{Cryoscope}

\begin{frame}{Flux pulse reconstruction}
  Transmon flux dependence:
  \begin{equation*}
    f_q(\Phi_q) \approx \left( \sqrt{8E_J E_C \left| \cos\left(\pi \frac{\Phi_q}{\Phi_0}\right) \right|} \right)
  \end{equation*}

  \begin{figure}
    \includegraphics[width=0.35\textwidth]{figures/cryoscope_pulse.png}
  \end{figure}
\end{frame}

\begin{frame}{Impact on chevron plots}
  \begin{figure}
    \centering
    \includegraphics[width=\textwidth]{figures/B2B4_nofilter.png}
    \vfill
    \includegraphics[width=\textwidth]{figures/B2B4.png}
  \end{figure}
\end{frame}

%\begin{frame}[t,fragile]{Filter determination}
%  
%\end{frame}
%
%\begin{frame}[t,fragile]{Results}
%  
%\end{frame}

\section{Conclusions \& Outlooks}


\begin{frame}[t,standout]
\Large
Questions?
\end{frame}


\begin{frame}{References}
    \nocite{*}
    \bibliographystyle{plain}
    \bibliography{bibliography}
\end{frame}

\begin{frame}{What is for?}
  \begin{columns}
    \begin{column}{0.5\textwidth}
      \begin{itemize}
        \item \textbf{Simulation of quantum system:} "Nature isn't classical, dammit, and if you want to make a simulation of nature, you'd better make it quantum mechanical, and by golly it's a wonderful problem, because it doesn't look so easy"
                 %\href{https://link.springer.com/article/10.1007/BF02650179}{\faBook\,\, Richard Feynman, 1982, Simulating Physics with Computers}
        \item Optimization and modeling (finance, traffic, weather...)
        \item Quantum Algorithms 
        \item Quantum Machine Learning
      \end{itemize}        
      \end{column}
      \begin{column}{0.5\textwidth}
        \begin{center}
            \includegraphics[width=0.8\textwidth]{figures/feynmann.jpg}
        \end{center}
      \end{column}
  \end{columns}
\end{frame}

\begin{frame}{Qubit platforms}
  \begin{center}
      \includegraphics[height=0.82\textheight]{figures/platforms.png}
  \end{center}
\end{frame}

\begin{frame}[fragile]{Standard Randomized Benchmarking protocol}
    \begin{columns}
      \begin{column}{0.5\textwidth}
        RB protocol
      {\setbeamertemplate{enumerate items}[default]
       \begin{enumerate}[leftmargin=*, label=\arabic*.]
         \item Initialize the system in the ground state
         \item For each sequence length \( m \), draw a sequence of Clifford group elements
         \item Calculate the inverse gate 
         \item Measure sequence and inverse gate
         \item Repeat the process for multiple sequences of the same length while varying the length
       \end{enumerate}}
      \end{column}
      \begin{column}{0.5\textwidth}
        RB features
        \begin{itemize}
          \item robust to SPAM errors
          \item faster than state tomography
          \item hardware-agnostic 
        \end{itemize}
      \end{column}
    \end{columns}
\end{frame}

\begin{frame}{Clifford gates}
  \begin{columns}
    \begin{column}{0.45\textwidth}
      \begin{itemize}
        \item Special subset of quantum gates that map Pauli operators to Pauli operators under conjugation
        \item Clifford gates group is generated by $H$, $S$, $CNOT$ gates
        \item Quantum circuits that consist of only Clifford gates can be efficiently simulated with a classical computer (Gottesman–Knill theorem)
      \end{itemize}
    \end{column}
  \end{columns}
\end{frame}

\end{document}
