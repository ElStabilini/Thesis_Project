% % -*- coding:utf-8 -*-
\documentclass[aspectratio=169,10pt]{beamer}
\nonstopmode

\usepackage{pifont}
\usepackage{subcaption}
\usepackage{fontawesome}
\usepackage{graphicx}
\usepackage{url}
\usepackage{enumitem}
\usepackage{caption}
\usepackage[style=authoryear,backend=biber]{biblatex}
\addbibresource{bibliography.bib}

\captionsetup{skip=4pt}
\input{colors}

% \usepackage{beamerthememetropolis}
\usetheme[progressbar=frametitle]{metropolis}
\metroset{progressbar=none}
\newcommand{\themename}{\textbf{\textsc{metropolis}}\xspace}
\newcounter{mainframenumber}
\newcommand{\backupbegin}{
  \setcounter{mainframenumber}{\value{framenumber}}
}



\usepackage{xcolor}
\setbeamertemplate{navigation symbols}{}
\setbeamercolor{background canvas}{bg=white}

% Custom footline with frame numbers on the left and navigation symbols on the right
\setbeamertemplate{footline}{
  \leavevmode%
  \hbox{%
    \begin{beamercolorbox}[wd=.1\paperwidth,ht=2.5ex,dp=1.125ex,leftskip=1em,rightskip=1em]{author in head/foot}%
      \usebeamerfont{footline} \insertframenumber{} / 19
    \end{beamercolorbox}%
    \hfill
    \begin{beamercolorbox}[wd=.9\paperwidth,ht=2.5ex,dp=1.125ex,leftskip=1em,rightskip=1em]{author in head/foot}%
      \usebeamerfont{footline} %\insertnavigation{2cm}
    \end{beamercolorbox}%
  }%
  \vskip0pt%
}

%%%%%%%%%%%%%%%%%%%%%%%%%%%%%%%%%%%%%%%%%%%%%%%%%%%%%% PRESENTATION START %%%%%%%%%%%%%%%%%%%%%%%%%%%%%%%%%%%%%%%%%%%%%%%%%

\title{Development of an open-source calibration framework\\ for superconducting qubits}
\subtitle{Master's degree in Physics}
\author{Elisa Stabilini}
\institute{Università degli Studi di Milano - Department of Physics}
\titlegraphic{\hfill}
\date{July 4th 2025}

\begin{document}

\maketitle

\begin{frame}{Table of contents}
    \setbeamertemplate{section in toc}[sections numbered]
    \setbeamertemplate{subsection in toc}[subsections numbered]  
    \tableofcontents[hideallsubsections]
\end{frame}

\begin{frame}[t,fragile]{Qibo framework}
  \begin{center}
      \includegraphics[height=0.80\paperheight]{figures/qibo_ecosystem.png}
  \end{center}
\end{frame}



\section{Superconducting qubits}

\begin{frame}{Artificial atoms}
  \begin{columns}
    \begin{column}{0.6\textwidth}
      \centering
      \begin{itemize}
        \item Qubit: two level system
        \hspace{10 mm}
        \item Superconducting qubits: use Josephson Junctions\\ to build anharmonic oscillators
      \end{itemize}
      \begin{figure}
        \includegraphics[height=0.5\textheight]{figures/Transmon.png}
        \caption{DOI: 10.1109/MAP.2022.3176593}
      \end{figure}
    \end{column}
    \begin{column}{0.6\textwidth}
      \centering
      \includegraphics[width=0.5\textwidth]{figures/cryostat.png}
    \end{column}
  \end{columns}
\end{frame}

\begin{frame}{Qubit control}

  \begin{columns}
    \begin{column}{0.45\textwidth}
      Electromagnetic pulse applied to the qubit
      \begin{equation*}
        V_d(t) = A\varepsilon(t)\sin{(\omega_d t + \alpha)},
      \end{equation*}
    
      Qubit - electric field Hamiltonian (RWA)
      \begin{equation*}
        \hat{H} = -\frac{\hbar (\omega_q - \omega_d)}{2} \hat{\sigma}_z + \frac{\hbar \Omega}{2} \varepsilon(t) \left( \hat{\sigma}_x \cos \alpha + \hat{\sigma}_y \sin \alpha \right)
      \end{equation*}
    
      General qubit evolution under electromagnetic pulse:
      \begin{equation*}
        R_{\hat{n}(\alpha)}(\theta) = e^{-\frac{i}{2} \hat{n}(\alpha) \cdot \vec{\sigma} \theta} = e^{-\frac{i}{2} (\hat{\sigma}_x \cos \alpha + \hat{\sigma}_y \sin \alpha) \theta}
      \end{equation*}
    
      where $\theta = \Omega\int_{0}^{+\infty}\varepsilon(t')dt'$
    \end{column}
    \begin{column}{0.55\textwidth}
      \centering
      \begin{figure}
        \vspace{2mm}
        \includegraphics[width=0.8\textwidth]{figures/TransmonCircuit.png}
        \vfill
        \includegraphics[width=0.8\textwidth]{figures/TransmonBoard.png}
        \caption{DOI: 10.1109/MAP.2022.3176593}
      \end{figure}
    \end{column}
  \end{columns}   


\end{frame}

\begin{frame}{State readout}
  \begin{columns}
    \begin{column}{0.35\textwidth}
      Qubit - resonator Hamiltonian:
      \begin{equation*}
        \hat{H} = \hbar\omega_r\hat{a}\hat{a}^\dagger - \frac{\hbar\omega_{01}}{2}\hat{\sigma}_z + \hbar g(\hat{\sigma}^+\hat{a}+\hat{\sigma}^-\hat{a}^\dagger)
      \end{equation*}\\
      \vspace{1.5em}
      Dispersive regime ($g \ll \omega_q - \omega_r$):
      \begin{equation*}
        \hat{H}_{disp} = \hbar(\omega_r - \chi\hat{\sigma}_z)\hat{a}^\dagger\hat{a} - \frac{\hbar}{2}(\omega_{01}+\chi)\hat{\sigma}_z
      \end{equation*}
      dispersive shift: $\chi = \frac{g^2}{\Delta},$ \hfill $\Delta = \omega_q - \omega_r$
    \end{column}
    \begin{column}{0.65\textwidth}
      \centering
      \includegraphics[width=0.75\textwidth]{figures/disp_sihft.png}
      \hspace{10mm}
      \includegraphics[height=0.4\textwidth]{figures/classification.png}
    \end{column}
  \end{columns}
\end{frame}

\begin{frame}{Superconducting qubit calibration}
  
  \begin{figure}
    \centering
    \includegraphics[width=0.9\textwidth]{figures/calibration.png}
  \end{figure}
  \vspace{1.5em}
  \begin{itemize}
    \item We need to calibrate \textit{only} native gates
    \item Native single qubit gates on superconducting qubits: $R_X(\pi)$, $R_X(\frac{\pi}{2})$, $R_Z(\theta)$
    \item For each pulse we calibrate frequency, duration, power, shape
  \end{itemize}
\end{frame}





\section{Average Clifford gate fidelity optimization}

\begin{frame}[t,fragile]{Randomized Benchmarking}

  \begin{itemize}
    \item Randomized benchmarking estimates average gate fidelity by applying random sequences of Clifford gates followed by an inverting gate.
    \item Randomization with Clifford gates provides a depolarized noise channel: $\rho \rightarrow d \frac{\mathbb{I}}{2} + (1-d)\rho$
  \end{itemize}

  \vspace{3mm}
  Randomized Benchmarking protocol:
  {\setbeamertemplate{enumerate items}[default]
   \begin{enumerate}[leftmargin=*, label=\arabic*.]
     \item Initialize the system in the ground state
     \item For each sequence length $m$ draw a sequence of Clifford group elements
     \item Calculate the inverse gate 
     \item Measure sequence and inverse gate
     \item Repeat the process for multiple sequences of the same length while varying the length
  \end{enumerate}}
\end{frame}

\begin{frame}[t,fragile]{Randomized Benchmarking}
  \begin{itemize}
    \item The survival probability decays exponentially with the number of Clifford gates $F(m) = Ap^m + B$ where $1-p$ is the depolarization rate.
    \item From $p$ we can extract the average error per Clifford gate
  \end{itemize}
  \begin{center}
    \vspace{0.5em}
    \includegraphics[width=\textwidth]{figures/rb.png}
    \vspace{1.25em}
    Can we optimize the average Clifford gate fidelity to automate $R_X(\pi)$ gate recalibration? 
  \end{center}
\end{frame}

\begin{frame}[t,fragile]{RB optimization [\cite{kelly_optimal_2014}]}
  \begin{center}
    Test closed-loop optimization with modern optimization libraries
    \vspace{1.25em}
    \includegraphics[width=0.9\textwidth]{figures/opt_workflow.png}
  \end{center}
  \vspace{1.25em}
  Optimization parameters:
  \begin{itemize}
    \item Pulse amplitude
    \item Pulse frequency
    \item Pulse shape (through $\beta$ DRAG parameter)
  \end{itemize}
\end{frame}


\begin{frame}[t,fragile]{Average Clifford gate Fidelity}
    \vbox to \textheight{ 
    \vspace{1em}
    \begin{minipage}[t]{\textwidth}
      \centering
      \includegraphics[height=0.35\textheight]{figures/NM_fid.png}
      \hspace{10mm}
      \includegraphics[height=0.35\textheight]{figures/fidelity_CMA.png}
    \end{minipage}
    \vspace{0.5em}
    \begin{minipage}[b]{\textwidth}
      \centering
      \includegraphics[width=0.85\textwidth]{figures/optuna.png}
    \end{minipage}
    \vspace{1em}
  }
\end{frame}

\begin{frame}[t,fragile]{Parameters evolution}
  \begin{figure}
    \begin{subfigure}[t]{0.315\textwidth}
      \includegraphics[width=\textwidth]{figures/amplitude.png}
    \end{subfigure}
        \begin{subfigure}[t]{0.315\textwidth}
      \includegraphics[width=\textwidth]{figures/frequency.png}
    \end{subfigure}
        \begin{subfigure}[t]{0.315\textwidth}
      \includegraphics[width=\textwidth]{figures/beta.png}
    \end{subfigure}

  \end{figure}
    \begin{figure}
    \begin{subfigure}[t]{0.315\textwidth}
      \includegraphics[width=\textwidth]{figures/CMA_amplitude.png}
    \end{subfigure}
        \begin{subfigure}[t]{0.315\textwidth}
      \includegraphics[width=\textwidth]{figures/CMA_frequency.png}
    \end{subfigure}
        \begin{subfigure}[t]{0.315\textwidth}
      \includegraphics[width=\textwidth]{figures/beta_CMA.png}
    \end{subfigure}
  \end{figure}
\end{frame}




\section{Library additions}
\subsection{Native RX90}

\begin{frame}{Native RX90}
  \begin{itemize}
    \item Qibolab single-qubit native gates are $R_X(\pi)$ and $MZ$
    \item $R_X(\frac{\pi}{2})$ gate is implemented by halving the calibrated values for $R_X(\pi)$ 
    \item Add native $R_X(\frac{\pi}{2})$ for more precise gates implementation
  \end{itemize}
  \begin{figure}
    \centering
    \includegraphics[width=\textwidth]{figures/RX90.png}
  \end{figure}
\end{frame}

\begin{frame}{Rabi amplitude experiment}
    \begin{figure}
    \centering
    \includegraphics[width=\textwidth]{figures/B4.png}
    \vfill
    \includegraphics[width=\textwidth]{figures/B4_90.png}
  \end{figure}
\end{frame}

\subsection{Cryoscope}

\begin{frame}{Flux pulse reconstruction [\cite{rol_time-domain_2020}]}
  \begin{columns}
    \begin{column}{0.4\textwidth}
      \centering
      Transmon frquency dependence on magnetic flux:
      \begin{equation*}
        f_q(\Phi_q) \approx \left( \sqrt{8E_J E_C \left| \cos\left(\pi \frac{\Phi_q}{\Phi_0}\right) \right|} \right)
      \end{equation*}\\
      \vspace{1em}
      Detuning as function of the flux pulse:
      \begin{align*}
        \Delta f_q = &\frac{\varphi_{\tau+\Delta\tau}-\varphi_\tau}{2\pi} \approx \\
        &\frac{1}{\Delta \tau} \int_{\tau}^{\tau + \Delta \tau} \Delta f_q(\Phi_{q, \tau + \Delta \tau} (t))
      \end{align*}
    \end{column}
    \begin{column}{0.6\textwidth}
      \begin{figure}
        \centering
        \includegraphics[width=0.85\textwidth]{figures/cryoscope_pulse.png}\\
        \vspace{3em}
        \includegraphics[width=0.85\textwidth]{figures/BlochEvolution.png}
        \caption{DOI: 10.1039/D2TC01258H}
      \end{figure}
    \end{column}
  \end{columns}
\end{frame}

\begin{frame}{Filter determination}
  \begin{figure}
    \centering
    \includegraphics[width=\textwidth]{figures/B4_ringin.png}
  \end{figure}
  \vspace{0.5em}
  {\setbeamertemplate{enumerate items}[default]
    \begin{enumerate}[leftmargin=*, label=\arabic*.]
      \item Determine exponential correction
      \item Obtain coefficients for the Infinite Impulse Response from exponential correction
      \item Determine Finite Impulse Response coefficients
      \item Obtain coefficients for real-time correction
    \end{enumerate}}
\end{frame}

\begin{frame}{Impact of correction on chevron plots}
  \begin{figure}
    \centering
    \includegraphics[width=\textwidth]{figures/B2B4_nofilter.png}
    \vfill
    \includegraphics[width=\textwidth]{figures/B2B4.png}
  \end{figure}
\end{frame}

\begin{frame}{Cryoscope routine}
  \centering
  \includegraphics[height=\textheight]{figures/cryoscope_routine.png}
\end{frame}


\section{Conclusions \& Outlooks}

\begin{frame}{Gate recalibration by RB optimization}
  \begin{itemize}
    \item[\ding{51}] We can always find parameters configuration with average Clifford gate fidelity > $99.5\%$
    \item[\ding{51}] Does not rely on manual tuning by an operator
    \item[\ding{55}] RB evaluation is computationally expensive ($\sim 30$ minutes)
    \item[\ding{55}] More stable methods (eg. Nelder-Mead) require many cost function evaluation
    \item[\ding{55}] Parameters drift makes optimization unstable
  \end{itemize}
  Possible future work: optimize RB parameters to allow a faster and more reliable optimization
\end{frame}

\begin{frame}{Library additions}
  \begin{itemize}
    \item[\ding{51}] Extended Qibolab and Qibocal libraries to support native $R_X(\pi/2)$ gates with dedicated calibration routines
    \item[\ding{51}] Implemented and added the Cryoscope calibration experiment to Qibocal library to correct flux-pulse distortions (average NMSE improvement $\sim 70\%$)
    \item[\ding{55}] Study long-time distortions of the flux-pulse
  \end{itemize}
  Possible future extensions: 
  \begin{itemize}[label={\raisebox{0.2ex}{\tiny$\bullet$}}]
    \item Readout optimization protocols
    \item Active qubit reset schemes
    \item Implement leakage mitigation strategies
  \end{itemize}
\end{frame}

\begin{frame}[t,standout]
\Large
Thank you
\end{frame}

%%%%%%%%%%%%%%%%%%%%%%%%%%%%%%%%%%%%%%%%%%% BACKUP SLIDES %%%%%%%%%%%%%%%%%%%%%%%%%%%%%%%%%%%%%%%%%%%

\backupbegin
\appendix

\begin{frame}{References}
    \printbibliography
\end{frame}


\section*{Backup slides}

\begin{frame}{What is for?}
  \begin{columns}
    \begin{column}{0.5\textwidth}
      \begin{itemize}[label=\textbullet]
        \item \textbf{Simulation of quantum system:} "Nature isn't classical, dammit, and if you want to make a simulation of nature, you'd better make it quantum mechanical, and by golly it's a wonderful problem, because it doesn't look so easy"\\
                 - Richard Feynman, 1982, Simulating Physics with Computers
        \item Optimization and modeling (chemistry, finance, traffic, weather...), eg. VQE, QAOA
        \item Quantum Algorithms 
        \item Quantum Machine Learning
      \end{itemize}
      \end{column}
      \begin{column}{0.5\textwidth}
        \begin{center}
            \includegraphics[width=0.8\textwidth]{figures/feynmann.jpg}
        \end{center}
      \end{column}
  \end{columns}
\end{frame}

\begin{frame}{Qubit platforms}
  \begin{center}
      \includegraphics[height=0.82\textheight]{figures/platforms.png}
  \end{center}
\end{frame}

\begin{frame}{Clifford gates}
  \begin{columns}
    \begin{column}{0.4\textwidth}
      \begin{itemize}[label=\textbullet]
        \item Special subset of quantum gates that map Pauli operators to Pauli operators under conjugation
        \hspace{10mm}
        \item Clifford gates group is generated by $H$, $S$, ($CNOT$) gates
        \hspace{10mm}
        \item Quantum circuits that consist of only Clifford gates can be efficiently simulated with a classical computer (Gottesman–Knill theorem)
      \end{itemize}
    \end{column}
    \begin{column}{0.6\textwidth}
      \begin{figure}
        \centering
        \includegraphics[width=\textwidth]{figures/HS.png}
        \includegraphics[width=\textwidth]{figures/CNOT.png}
        \vfill
      \end{figure}
    \end{column}
  \end{columns}
\end{frame}



\begin{frame}{Calibration workflow \& assessment}
  \vbox to \textheight{ 
    \begin{minipage}[t]{\textwidth}
      \centering
      \begin{columns}
        \begin{column}{0.4\textwidth}
          Procedure:
          \setbeamertemplate{enumerate items}[default]
          \begin{enumerate}[leftmargin=*, label=\arabic*.]
            \item Resonator characterization
            \item Qubit characterization
            \item Gate calibration
            \item Gate set characterization
          \end{enumerate}
        \end{column}
      
        \begin{column}{0.4\textwidth}
          Metrics example:
          \begin{itemize}[label={\raisebox{0.2ex}{\tiny$\bullet$}}]
            \item readout \& assignment fidelity
            \item relaxation time $T_1$
            \item decoherence time $T_22$
            \item gate fidelity
          \end{itemize}   
        \end{column}
      \end{columns}
    \end{minipage}
    \vspace{1em}
  }
\end{frame}

\begin{frame}{Calibration workflow \& assessment}
  \vbox to \textheight{ 

    \begin{minipage}[t]{\textwidth}
      \centering
      \begin{columns}
        \begin{column}{0.4\textwidth}
          Main steps:
          \setbeamertemplate{enumerate items}[default]
          \begin{enumerate}[leftmargin=*, label=\arabic*.]
            \item Resonator characterization
            \item Qubit characterization
            \item Gate calibration
            \item Gate set characterization
          \end{enumerate}
        \end{column}
      
        \begin{column}{0.4\textwidth}
          Calibration quality metrics:
          \begin{itemize}[label={\raisebox{0.2ex}{\tiny$\bullet$}}]
            \item readout \& assignment fidelity
            \item relaxation time $T_1$
            \item decoherence time $T_22$
            \item gate fidelity
          \end{itemize}   
        \end{column}
      \end{columns}
    \end{minipage}
    \vspace{1em}


    \begin{minipage}[b]{\textwidth}
      \centering
      \includegraphics[height=0.4\textheight]{figures/cal_results.png}
    \end{minipage}
  }
\end{frame}

\begin{frame}{RB optimization summary}
  
\end{frame}


\begin{frame}{Filter determination details}
  
\end{frame}



\end{document}
